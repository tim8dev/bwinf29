\section{Aufgabe 3: ``Logistisch''}
\subsection{Lösungsidee}
\subsubsection{Teilaufgabe 1}
Man betrachte für jeden Standort A, B und C die Fahrzeuge, die an jedem der fünf Tage dort stehen sollen.
Eine negative Anzahl Fahrzeuge singnalisiert dabei einen entsprechenden bedarf an Fahrzeugen.
Nun rechnet man den Bedarf an Fahrzeugen für jeden Tag und Standort wie folgt aus:
Seien zu Beginn der Woche an jedem Standort 0 Fahrzeuge.
Anhand des Tourenplans, an dem ablesbar ist, wieviele Fahrzeuge von welchem Standort zu welchem fahren,
lässt sich die Anzahl der erforderlichen Fahrzeuge des nächsten Tages berechnen. So werden auch alle Folgetage berechnet.
Für jeden Standort wird nun der Tag gesucht an dem der Bedarf an Fahrzeugen am höchsten ist (die Zahl obiger rechnung also am kleinsten, ``negativsten'' ist).
Ist dieser errechnete Wert negativ, so ist der Betrag dieses Wertes die Startanzahl, ist er jedoch positiv, genügt der Startwert 0.
Nachdem dies für jeden Standort erfolgt ist, hat man die Startanzahlen für alle Standorte A, B und C.
\subsubsection{Teilaufgabe 2}
(Zu dieser Teilaufgabe 2 wurden zwar Überlegungen angestellt, für eine Implementierung fehlte jedoch die Zeit.)
Um nur eine möglichst niedrige Anzahl Fahrzeuge bereitstellen zu müssen, 
können Leerfahrten benutzt werden, um überschüssige Fahrzeuge an einem Standort, zu einem anderem zu bringen.
Hier kann die in Teil 1 erzeugte Liste für jeden Standort mit theoretischen - also auch negativen - Werten benutzt werden.
Wenn an einem Standort negative Werte auftauchen, wird versucht von den beiden anderen Standorten Fahrzeuge durch Leerfahrten bereitzustellen.
Da es Ziel ist, den minimalen negativen Wert auszugleichen, sollten immer an dem Tag mit dem minimalen Wert Optimierungsversuche erfolgen.
Nach einem Optimierungsversuch werden die Listen neu berechnet und bei dem neuem Minimum weiteroptimiert.
Wichtig ist, dass eine Leerfahrt nur dann erfolgt, wenn dieses Fahrzeug nicht nächsten Tag am derzeitigen Standort gebraucht wird.
\subsection{Programm-Dokumentation}
Aus den Vorüberlegung folgt:
Die Startanzahl der Fahrzeuge für einen Standort ist der Betrag des Minimums aller kalkulierten Fahrzeuganzahlen an einem jedem Tag und dem Startwert 0.
\paragraph{Implementierung:} Nach dem Einlesen der Daten aus einer Datei, werden zunächst die Differenzen zwischen den Ankünften und Abfahrten
von Fahrzeugen berechnen. Diese Werte errechnen sich aus der Summe der Zufahrten von den beiden anderen Standorten
minus die Summe der Abfahrten zu den anderen beiden Standorten.
Bei mehr Ab- als Zufahrten ist der Differenzwert somit negativ. 
Da die kalkulierte Fahrzeuganzahl eines Tages sich aus der Addition des Differenzwertes mit dem Vortageswert errechnet, können
Listen mit den Fahrzeuganzahlen rekursiv an dem ersten Tag beginnend berechnet werden
(durch eine Hilfsfunktion ``berechneStandort'' und dem theoretischem Startwert 0).
Nachdem an jede Liste eine 0 (Startwert) angehängt wurde, wird der Betrag des Minimums dieser Liste als tatsächlich benötigter
Startwert für diesen Standort zurückgegeben.
(Der Startwert 0 wird angehängt, um ein nicht positives Minimum zu erhalten.)
\paragraph{Testfälle:} Es wurde eine Datei TestCases.scala erstellt, in der ein ``jUnit TestCase'' implementiert ist.
Dieser enthält verschiedene Eingaben die getestet werden, unter anderem die Mustereingaben aus der Materialsektion von der Webseite des BWINFs.
\subsection{Programm-Ablaufprotokoll}
Ablauf mit einer Mustereingabe (die anderen wurden ebenfalls getestet, dies ist hier jedoch nicht abgebildet):
\lstset{language=Python}
\lstset{basicstyle=\footnotesize}
\begin{lstlisting}
# A->B A->C B->A B->C C->A C->B
1 2 3 4 5 6
2 3 4 5 6 1
3 4 5 6 7 8
1 2 3 4 5 6
1 2 3 4 5 6
4 3 5 6 1 2
\end{lstlisting}
Wird eingelesen und folgt zu diesem Ablaufprotokoll:
\lstset{language=Scala}
\lstset{basicstyle=\footnotesize}
\begin{lstlisting}
Fahrzeugdifferenzen für Standort A: 
  5
  5
  5
  5
  5
  -1
Fahrzeugdifferenzen für Standort B: 
  0
  -6
  0
  0
  0
  -5
Fahrzeugdifferenzen für Standort C: 
  -5
  1
  -5
  -5
  -5
  6
 Standort A: aktueller Zwischenwert: 5
 Standort A: aktueller Zwischenwert: 10
 Standort A: aktueller Zwischenwert: 15
 Standort A: aktueller Zwischenwert: 20
 Standort A: aktueller Zwischenwert: 25
 Standort A: aktueller Zwischenwert: 24
 Standort B: aktueller Zwischenwert: 0
 Standort B: aktueller Zwischenwert: -6
 Standort B: aktueller Zwischenwert: -6
 Standort B: aktueller Zwischenwert: -6
 Standort B: aktueller Zwischenwert: -6
 Standort B: aktueller Zwischenwert: -11
 Standort C: aktueller Zwischenwert: -5
 Standort C: aktueller Zwischenwert: -4
 Standort C: aktueller Zwischenwert: -9
 Standort C: aktueller Zwischenwert: -14
 Standort C: aktueller Zwischenwert: -19
 Standort C: aktueller Zwischenwert: -13
Anzahl für Standort A: 0
Anzahl für Standort B: 11
Anzahl für Standort C: 19
\end{lstlisting}

\subsection{Programm-Text}
Der Code ist entnommen aus Aufgabe3/src/tau/tim/logistik/TourenPlan.scala auf der CD.
Zum Ausführen ist zwar noch zusätzlich Code erforderlich, wie z.B. der Dateieinlesecode, jedoch wird hier nur der inhaltlich relevante Teil eingefügt.
\lstset{language=Scala}
\lstset{basicstyle=\footnotesize}
\begin{lstlisting}
 /**
   * Berechnet aus einer Liste mit je 6 Werten eine mögliche Zahl an Fahrzeugen
   * für jeden Standort A, B, und C.
   * Die Wert in der Liste sind wie in den Beispieldateien:
   * A->B, A->C, B->A, B->C, C->A, C->B
   *
   * @param einAus: Die Liste mit den Werten.
   * @param log: Eine Funktion zur Ausgabe des Protokolls
   */
  def berechne(einAus: List[(Int,Int,Int,Int,Int,Int)], log: String => _) = {
    // Aus den Spalten entsprechende Werte einlesen
    // Berechnet gleichzeitig die Differenz zwischen Ein- und Ausfuhren für jeden Tag.
    val a_diff = for(i <- einAus) yield i._3+i._5 - (i._1+i._2)
    val b_diff = for(i <- einAus) yield i._1+i._6 - (i._3+i._4)
    val c_diff = for(i <- einAus) yield i._2+i._4 - (i._5+i._6)

    log { "Fahrzeugdifferenzen für Standort A: " + a_diff.mkString("\n  ", "\n  ", "") }
    log { "Fahrzeugdifferenzen für Standort B: " + b_diff.mkString("\n  ", "\n  ", "") }
    log { "Fahrzeugdifferenzen für Standort C: " + c_diff.mkString("\n  ", "\n  ", "") }

    // Die Fahrzeugzahlen für jeden Standort:
    val a = 0 :: berechneStandort(0, a_diff, (s:String) => log{ " Standort A: " + s})
    val b = 0 :: berechneStandort(0, b_diff, (s:String) => log{ " Standort B: " + s})
    val c = 0 :: berechneStandort(0, c_diff, (s:String) => log{ " Standort C: " + s})

    // Die Anzahl der Fahrzeuge für jeden Standort ist das Minimum.
    val res = (-a.min, -b.min, -c.min) // Minimum <= 0, weil 0 am Anfang angehängt wurde.
    log { "Anzahl für Standort A: " + res._1 }
    log { "Anzahl für Standort B: " + res._2 }
    log { "Anzahl für Standort C: " + res._3 }
    res
  }

 private def berechneStandort(vorher : Int, diffs: Seq[Int], log: String => _) : List[Int] =
    diffs.headOption match {
      case Some(diff) =>
        // Der jetzige Wert ist der vorherige plus die Differenz aus den Ein- und Ausfuhren
        val jetzt = vorher + diff
        log { "aktueller Zwischenwert: " + jetzt }
        jetzt ::
        // Rekursiver Aufruf mit Zwischenwert jetzt und Restwerten diffs.tail
        berechneStandort(jetzt, diffs.tail, log)
      case _ => Nil // wenn keine Werte mehr vorhanden, Rekursionsschluss.
    }
\end{lstlisting}

\subsection{Programm-Nutzung}
% Include Scala lib in Distribution.
Das Programm kann durch den Befehl `java -jar Aufgabe3/dist/Logistik.jar' ausgeführt werden (Voraussetzungen und Startanleitung siehe \ref{Vor}).
Die Eingabedaten sind in der Datei Aufgabe3/dist/table.txt. Diese wird zu Programmbeginn eingelesen.
Um diese Mustereingabe zu ändern, muss der gesamte Aufgabe3 Ordner in ein beschreibbares Verzeichnis kopiert werden. Geben Sie dann Ihre gewünschte
Eingabe in die Aufgabe3/dist/table.txt Datei. Führen Sie den Befehl statt auf dem Wurzelverzeichnis der CD dann auf dem entprechendem Verzeichnis aus.
 