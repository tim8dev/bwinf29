\lstset{language=Scala}
\chapter[Erste bearbeitete Aufgabe: (1) Kisten in Kisten in Kisten]{Erste bearbeitete Aufgabe: \newline (1) Kisten in Kisten in Kisten}
\minitoc
\clearpage
\section{Lösungsidee}
 Um Frau Y. zu helfen, soll ein Algorithmus entwickelt werden, der zu gebenen Kisten diese so ineinander packt,
 dass die äußersten Kisten in Summe ein möglichst geringes Volumen vorweisen.
 Meine Grundidee ist es, die Schachtelung der Kisten als binären Baum aufzufassen.

 Optimal wäre es natürlich, wenn die äußeren Kisten zusammen gesehen ein minimales Außenvolumen bieten würden.
 Dies kann in der Tat erreicht werden, wenn alle möglichen Schachtelungen berechnet werden und dann die mit dem kleinsten Außenvolumen zurückgegeben wird.
 Um jedoch nicht für jede Kiste nacheinander für jede andere zu überprüfen ob sie hineinpasst, wird eine klare Reihenfolge, in der die Kisten ineinander gepackt werden sollen, gesucht.
 Dies kann erreicht werden, indem die Kisten dem Volumen nach von groß nach klein sortiert werden.
 Anschließend werden - beginnend bei der größten - alle Kisten nacheinander zu einer Schachtelung kombiniert.
 Diesen Ansatz nenne ich Brute-force-Algorithmus und beschreibe ihn in \ref{bruteforce} genauer.
 Ohne große Überlegungen lässt sich feststellen, dass solch ein Ansatz prinzipiell immer ein Optimum liefert.
 Problematisch wird es allerdings, wenn die Laufzeit in Betracht gezogen wird, die zum Berechnen benötigt wird.
 Denn, wie wir später noch genauer sehen werden, lassen sich bereits Probleme mit kleiner Anzahl Kisten nicht mehr in akzeptabler Zeit lösen.
 Deswegen betrachten wir nachfolgend noch weitere Ansätze und Algorithmen, die deutlich besser skalieren.
\section{Algorithmen und Datenstrukturen}
\subsection{Anmerkung zur Notation}
 Vor der folgenden, eigentlichen Dokumentation möchte ich noch von mir verwendete Konventionen erläutern.
 Betrachte ich in einem Teil der nachfolgenden Dokumentation und Erläuterung Code, bzw. Codeausschnitte,
 benutze ich |monospaced Font| sowie die im Code benutzten Bezeichner, um diese darzustellen.

\subsection{Datenstrukturen}
 Um das Problem präzise bearbeiten zu können, möchte ich die von mir verwendeten Datenstrukturen - zum Teil auch mathematisch - beschreiben.

 Als einen \emph{Kistenbaum} bezeichne ich einen binären Baum bei dem jeder Knoten eine Kiste repräsentiert.
 In diesem gilt für jeden Knoten $N$ und die durch ihn repräsentierte Kiste $K$, dass die Nachfolgerkisten,
 die durch die null, ein oder zwei Nachfolgerknoten von $N$ repräsentiert werden, gleichzeitig in $K$ passen.
 Ein Kistenbaum stellt also eine äußere Kiste sowie alle enthaltenen verschachtelten Kisten dar.

 Als einen \emph{Kistensatz} bezeichne ich eine Multimenge von Kistenbäumen.
 Ein Kistensatz kann also eine Lösung im Sinne der Aufgabenstellung darstellen.

 Das Grundproblem ist nun, zu gegebenen Kisten $k_1,k_2,\dots,k_n$
 einen Kistensatz $ks$ mit den Wurzeln $w_1,\dots,w_m$ seiner Kistenbäume zu erzeugen, wobei alle Kisten $k_1,k_2,\dots,k_n$ genau einmal enthalten sind.
 Seien $v_1,\dots,v_m$ die jeweiligen Volumina der Wurzeln $w_1,\dots,w_m$. Dann gilt es als weitere Aufgabe $\sum_{i=0}^{m}{v_i}$ zu minimieren.

 Die Kisten $k_1,k_2,\dots,k_n$ werden als Liste aufgefasst, eine Menge ist ungeeignet da Kisten (mit gleichen Maßen) mehrfach vorkommen können.
\subsection{Algorithmen}
 In den nachfolgenden Abschnitten werden insgesamt drei Algorithmen vorgestellt.
 Der erste im Abschnitt \ref{bruteforce} liefert zwar optimale Ergebnisse, allerdings ist seine Laufzeitkomplexität unpraktikabel hoch.
 Es wird deswegen ein Trade-off zwischen Laufzeit und Güte gemacht.
 Hierzu habe ich zwei weitere Lösungsideen in den Abschnitten \ref{auf} und \ref{ink} erstellt, die unterschiedliche Kompromisse darstellen.
 Es ist offensichtlich, dass Optimalität dadurch nicht erreicht wird.
 Die grundlegende Motivation ist nun,
 das Streben nach Optimalität aufzugeben und stattdessen in menschlicher Zeit\footnote{menschliche Zeit $<<$ Lebenserwartung eines Menschen ($75$ Jahre)}
  trotzdem eine gute Packung auch zu einer großen Anzahl von Kisten zu finden.
 Anders gesagt, wird eine im Allgemeinen suboptimale, aber trotzdem brauchbare und vor allem nach praktischen Maßstäben (Anzahl Kisten, Rechenzeit) sinnvolle Lösung gefunden.
\subsection{Brute-force-Algorithmus}
\label{bruteforce}
\subsubsection{Lösungsidee}
\lstset{basicstyle=\ttfamily}
 Der oben bereits skizzierte Algorithmus soll nun präzisiert werden.
 Dieser Algorithmus |OptimalPacker| hat drei Teile.
\begin{enumerate}
 \item Die Liste wird entsprechend dem Volumen von groß nach klein sortiert.
  Das heißt, dass jede Kiste $k$ an der Stelle $i$, in keine Nachfolgerkiste $k_{i+1}, \dots, k_n$ passt,
   weil eine Kiste nur dann in eine andere Kiste passen kann, wenn ihr Volumen kleiner ist.
  Dies wiederum ermöglicht die Kombination ``entlang'' der Liste.
  So müssen andere Reihenfolgen von Kisten nicht berücksichtigt werden, was den Algorithmus erheblich beschleunigt.
 \item Die resultierende Liste der Kisten wird nacheinander zu Kistensätzen kombiniert.
  Die Hilfsfunktion |packSchritt| erzeugt aus einer Menge von Kistensätzen durch Hinzufügen einer gegebenen Kiste die Menge aller möglichen Kistensätze.
  Diese werden dann weiter mit der nächsten Kiste zu noch mehr Kistensätzen kombiniert.
  Am Ende sind alle Elemente der Eingabeliste abgearbeitet.
 \item Nun wird der Kistensatz ausgewählt, welcher das kleinste Außenvolumen hat und zurückgegeben.
\end{enumerate}
 Wichtig ist, um den Code zu verstehen, dass |(satz ++< kiste)| alle Möglichkeiten erzeugt,
 wie man die Kiste in einen |Kistensatz| einfügen kann.\footnote{Nähere Erläuterungen dazu später, für den Algorithmus ist dies nicht direkt relevant.}

\lstset{basicstyle=\ttfamily\small}
\begin{lstlisting}
def min(kistenListe: Seq[KisteLeer]) = {
  val kistenSätze = packe(kistenListe)  // Teil 1&2: siehe packe
  kistenSätze min                      //  Teil 3: Minimum (nach Volumen)
   Kistensatz.Ordnung.nachVolumen
}

def packe(kistenListe: Seq[KisteLeer]) = {
  val kisten = sortiere(kistenListe) // Teil 1: Sortieren
  /* Teil 2:
   * Nacheinander für jede Kiste und jeden bereits erzeugten Kistensatz alle
   * möglichen Schachtelungskombiniationen mit dieser Kiste erzeugen. */
  kisten.foldLeft(Set[Kistensatz]()) ( packSchritt )
}

def packSchritt(sätze: Set[Kistensatz], kiste: KisteLeer) =
  if(sätze.isEmpty)
    Set(Kistensatz(kiste :: Nil)) // KistenSatz nur mit der Kiste
  else for { satz <- sätze 		  // Für jeden satz in sätze
	     neuerSatz <- satz ++< kiste // erzeuge neue Möglichkeiten
           } yield neuerSatz            // und gib diese zurück
\end{lstlisting}
\lstset{basicstyle=\ttfamily}
\subsubsection{Laufzeitverhalten}
 Das Laufzeitverhalten dieses Algorithmus' ist fatal. Es muss im letzten Schritt eine Kiste in bis zu $(n-1)!$ Kistensätze gepackt werden.
 Die Laufzeit, eine Kiste in einen Kistensatz zu packen, ist $O(n)$.
 Dies gilt, da jede Kiste $k$ des bestehenden Kistensatzes bis zu $n$ Möglichkeiten bietet, die neue Kiste in $k$ einzupacken.
 Wir erhalten also $n! + n \cdot (n-1)! +  (n-1) \cdot (n-2)! + \dots + 2 \cdot 1! + 1$. Sprich $O(n \cdot n!)$.

 Auch wenn der Worstcase meist nicht erreicht wird, beispielsweise wenn es für 40 Kisten eher $20!$ Möglichkeiten gibt,
 würde die Berechnung aller Möglichkeiten bereits $8 \cdot 10^{15}$ Jahre benötigen,
  unter der Annahme, dass das Prüfen und Hinzufügen einer Kiste in eine andere Kiste eine Nanosekunde dauert.

 Unter der gleichen Annahme können etwa $15!$ Operationen in einer Stunde ausgeführt werden.
 Unter der Annahme, dass es etwa $x!$ Möglichkeiten für $2x$ Kisten gibt.
 können $2 \cdot (15-1)=28$ Kisten in einer Stunde in allen Möglichkeiten gepackt werden.
\subsubsection{(Un)mögliche Parallelisierung}
 Eine Überlegung ist, das Laufzeitverhalten durch Parallelisierung zu verkürzen.
 Allerdings verspricht dies aufgrund der hohen Laufzeitkomplexität von $O(n \cdot n!)$ kaum Abhilfe.
 Selbst bei 100 Prozessoren, sprich einer hundertfachen Beschleunigung\footnote{Eine solche Beschleunigung wird in der Praxis nie erreicht,
 es muss immer ein gewisser Overhead für Synchronisation und sequentielle Programmabläufe berücksichtigt werden.},
 können gerade mal $17!$ Operationen pro Stunde ausgeführt werden. Das entspricht $2 \cdot (17-1)=32$ Kisten.
 Es können also lediglich etwa $14\%$ ($32/28=1,1428\dots$) mehr Kisten gepackt werden, was nicht nennenswert ist.
\subsubsection{Problem der praktischen Nutzung}
 Frau Y. hat also ihre mittlerweile 28 Kisten optimal packen können. Dadurch ist nun Platz in ihrem Keller frei geworden.
 Allerdings empfindet sie es als widersprüchlich, dass sie genau deswegen nicht mehr Kisten in ihren Keller stellen kann, weil sie versucht eine optimale Lösung zu finden,
  was aber nur für kleine Kistenanzahlen gelingt.
 Wenn beispielsweise 200 Kisten gepackt werden sollen, bleiben 170 Kisten neben 30 optimal gepackten ungepackt.
\subsection{Brute-force-Algorithmus nach Aufteilung}
\label{auf}
\subsubsection{Lösungsidee}
 Eine Möglichkeit ist, den Brute-force-Algorithmus immer nur auf eine Gruppe,
  also einen Teil der Kisten, anzuwenden und danach diese so erzeugten Kistensätze nebeneinander zu stellen.
 
 Sei $t$ die maximale Anzahl Kisten pro Gruppe.
 Dann funktioniert diese Variante ``Brute-force-Al\-go\-rith\-mus nach Aufteilung'' so,
 dass der in \ref{bruteforce} beschriebene Algorithmus für jede Gruppe angewandt wird.

 Sei folgend $m$ die Anzahl Gruppen, also die Anzahl der Kisten geteilt durch $t$.
 Um die Kisten in Gruppen aufzuteilen, werden die Kisten einmal traversiert,
 dabei wird die erste Kiste der ersten Gruppe, die zweite Kiste der zweiten Gruppe, usw.\ und die $m.$ Kiste der $m.$ Gruppe.
 Dann werden die Gruppen wieder neu gezählt, also wird die $m+1.$ Kiste der ersten Gruppe, die $m+2.$ der zweiten, usw.\ zugeordnet.

 Nun soll der Algorithmus noch anhand von Scala-Code erläutert und präzisiert werden.
 Zunächst werden die Kistengruppen mit Hilfe der Funktion |teileKisten| berechnet (Z.\ 2).
 Diese Funktion folgt dem oben genanntem Prinzip, ist allerdings sehr technisch und wird daher ausgelassen.
 Dann wird für jede Kistengruppe der Algorithmus |OptimalPacker| aus \ref{bruteforce} aufgerufen (Z.\ 3 - 6).
 Schließlich werden die Kistensätze nebeneinander gestellt und zurückgegeben (Z.\ 8 - 10).
\begin{lstlisting}
def min(kisten: Seq[KisteLeer]): Kistensatz = {
  val kistengruppen = teileKisten(kisten)
  val kistensätze = for { // Für jede Kistengruppe
    kistengruppe <- kistengruppen
  } yield
     OptimalPacker min kistengruppe // packe nach Brute-force
  
  kistensätze reduceLeft {       // Kombiniere Kistensätze
    _ neben _                   // stelle diese nebeneinander
  }
}
\end{lstlisting}

\subsubsection{Laufzeitverhalten}
 Dieser Algorithmus ist bezüglich seiner Laufzeit streng polynomiell. Genauer gesagt, kann er sogar in linearer Zeit ausgeführt werden.
 Dies ergibt sich aus einer Laufzeitanalyse des Algorithmus.

 Sei im Folgenden $t$ der Teilungsfaktor, also die Anzahl Kisten die jeweils eine Gruppe bilden.
 Das Aufteilen der Kisten in Gruppen braucht $O(n)$. Man erhält also $\frac{n}{t}$ Gruppen.
 Eine dieser Gruppen zu packen, geschieht in konstanter Zeit. Auch wenn der Brute-force-Algorithmus ursprünglich eine Komplexität von $O(m \cdot m!)$ besitzt,
 ist mit $m=t$ seine Laufzeit $O(t \cdot t!) = O(1)$, also konstant bei konstantem $t$.
 Daraus ergibt sich letztendlich eine Laufzeitkomplexität für Gruppenbildung, Packen und Nebeneinanderstellen von
  \[ O(n + n + \frac{n}{t} \cdot 1) = O(n). \]
 Diese Laufzeitkomplexität ist sogar optimal, denn es muss in jedem Fall jede Kiste einmal betrachtet werden, spätestens bei der Ausgabe der Lösung.
\subsubsection{Parallelisierung}
 Dieser Algorithmus lässt sich schnell und einfach parallelisieren.
 Die Aufteilung lässt sich parallelisieren, indem $\frac{n}{t}$ parallele Prozesse jeweils immer das $t.$ Element der Liste der Kisten in eine neue anfügen.
 Diese Prozesse starten jeweils um eins versetzt.
 Sprich, der erste beim ersten Element, der zweite beim zweiten, usw., der $\frac{n}{t}.$ beim $\frac{n}{t}.$ Element.
 Nach der Aufteilung kann jeder Prozess unabhängig von einander den jeweils minimalen Kistensatz berechnen.
 Anschließend müssen diese zusammengefasst werden. Auch dieser Schritt lässt sich parallelisieren.
 In jedem Schritt werden die unmittelbar nebeneinander gestarteten Prozesse zu einem kombiniert, indem die minimalen Kistensätze kombiniert werden.
 Es sind so also $\log_2(\frac{n}{t})$ Schritte nötig. Der Parallelisierungsgrad nimmt in jedem Schritt um die Hälfte ab.

 Nun betrachten wird die benötigte Laufzeit für ein Problem der Größe $n$ auf einem System mit $p := \frac{n}{t}$ Prozessoren.
 Bis zum Kombinieren der Kistensätze laufen die Prozesse vollständig unabhängig voneinander.
 Also ist die Zeit hierfür gleich die des größten Einzelproblems, also konstant, wie oben dargestellt.
 Anschließend fällt der Parallelisierungsgrad mit jedem Kombinationsschritt.
 Jeder Kombinationsschritt kann in $O(1)$ berechnet werden, da wiederum alle Prozesse parallel arbeiten.
 (Ein Kombinationsschritt hat konstante Laufzeit.)
 Es sind jedoch $log_2(\frac{n}{t})$ Schritte notwendig.
 Die Gesamtlaufzeit ist also \[O(1)+O(\log_2(\frac{n}{t}))=O(\log(\frac{n}{t})=O(\log n).\]
 Zusammenfassend lässt sich folgendes sagen:
 \emph{$n$ Kisten lassen sich mit dem Algorithmus ``Brute-force nach Aufteilung'' auf einem System mit $\frac{n}{t}$ Prozessoren in $O(\log n)$ Zeit zu einem Kistensatz packen.}

 Für ein System mit 8 Prozessoren lassen sich bis zu $8 \cdot t = 120$ (mit $t=15$) Kisten in logarithmischer Zeit packen.
 Auf modernen Grafikkarten, die mehr als 1000 Prozessoren besitzen, lassen sich also bis zu $1000 \cdot 15=15000$ Kisten in logarithmischer Zeit packen.

 Leider ist dies nur ein theoretischer Wert. Denn nach dem Amdahlschem Gesetz\footnote{Goetz, Brian; Peierls, Tim; Bloch, Joshua; Bowbeer, Joseph; Holmes, David; LeaDoug: Java Concurrency in Practice, S.\ 225 ff., Addison-Wesley, 2006}
  gibt es immer einen Teil des Algorithmus' der sequentiell ausgeführt werden muss.
 Außerdem existiert - ebenfalls nach Amdahl\footnote{eben da} - ein gewisser Synchronisationsoverhead.
\subsection{Inkrementeller Algorithmus}
\label{ink}
\subsubsection{Lösungsidee}
 Ein etwas anderer Ansatz ist, Kistensätze inkrementell zu erzeugen.
 Das heißt, zu einem - mehr oder weniger gut - gepacktem Kistensatz und einer zu packenden Kiste soll genau ein neuer Kistensatz,
  der zusätzlich die neue Kiste enthält, geliefert werden.
 Dieser soll wiederum möglichst gut, also dicht gepackt sein.
 Eine weitere Beschränkung, die ich an den Algorithmus stelle, ist, dass er in höchstens $O(n)$ Zeit diesen neuen Kistensatz liefert.
 Mit solch einem Algorithmus lassen sich alle Kisten zusammen in $O(n \cdot n)=O(n^2)$ Zeit packen.
 Im Folgenden betrachten wir den Algorithmus zunächst als Online-Algorithmus, anschließend betrachten wir mögliche Strategien zum Hinzufügen einer Kiste
  und schließlich betrachten wir den Algorithmus als Offline-Algorithmus.
\subsubsection{Online-Algorithmus}
 Der oben skizzierte Algorithmus kann so genutzt werden, dass immer sofort die nächste Kiste in einen vorhandenen Kistensatz hinzugefügt wird.
 Der Algorithmus kann also von Frau Y.\ verwendet werden,
  um eine neue Kiste in den bestehenden Kistensatz im Keller hinzuzufügen, ohne alles neu berechnen zu müssen.
 Der Algorithmus erzeugt einen Kistensatz also unabhängig von später hinzuzufügende Kisten.
 Es handelt sich also um einen Online-Algorithmus. Die Entwicklung eines Online-Algorithmus' ist in der Aufgabenstellung weder explizit noch implizit gefordert.
 Dieser stellt also eine Erweiterung dar.
 Diese fand ich insofern sinnvoll, da sie ein Anwendungsfall direkt erfüllt, nämlich genau den, wenn Frau Y. eine einzelne Kiste erhält.
 Auf Code wird an dieser Stelle verzichtet. Im Prinzip ist dies wieder ein |foldLeft| über eine Liste, allerdings ist der restliche Code sehr technisch
  und wird daher ausgelassen.
\lstset{basicstyle=\ttfamily}
\subsubsection{Strategien}
\label{strats}
 Ein Algorithmus, der eine |Kiste| in linearer Zeit in einen |Kistensatz| packen kann, bezeichne ich als \emph{Strategie}.
 Es gibt eine Vielzahl an Strategien, recht naheliegend sind unter anderem folgende.

\begin{center}
\begin{tabular}{ll}
 \textbf{FindeHalbleeren}    & Sucht eine |KisteHalb|, die noch Platz für die neue |Kiste| bietet. \\
 \textbf{FindeGrößerenLeeren} & Sucht eine |KisteLeer|, die noch Platz für die neue |Kiste| bietet. \\
 \textbf{FindeZwischenraum}  & Sucht eine |Kiste|, die durch Entfernen einer Kind-|Kiste| genug Platz für \\
                             & die neue |Kiste| bietet, die wiederum die Kind-|Kiste| aufnehmen kann. \\
 \textbf{FindeKleinereWurzel}& Sucht eine Wurzel-|Kiste|, die in die neue Kiste passt.      \\
\end{tabular}
\end{center}

 Nun sollen die einzelnen Strategien noch genauer erläutert werden.
 Die Strategien suchen alle in allen Kistenbäumen eines Kistensatz mit Hilfe der Funktion |finde| die erste Kiste,
  die bestimmte Voraussetzungen erfüllt.
 Dabei wird jeder Baum traversiert. Treffen die Voraussetzungen auf eine bestimmte |Kiste| nicht zu,
  so werden alle in dieser |Kiste| gepackten Nachfolgerkisten traversiert.
 Nachfolgend bezeichne ich mit |kNeu| die einzufügende |Kiste|.
\paragraph{FindeHalbleeren}
 Diese Strategie sucht eine |Kiste| |kh|, die bereits einen Nachfolger hat und trotzdem Platz für |kNeu| bietet.
 Nachdem so eine |Kiste| gefunden wird, wird die |Kiste| |kNeu| in |kh| eingefügt.
\paragraph{FindeGrößerenLeeren}
 Diese Strategie verhält sich ähnlich wie |FindeHalbleeren|. Es wird jedoch eine Kiste |big| gesucht, die leer und größer als |kNeu| ist.
\paragraph{FindeZwischenraum}
 Dies ist die komplizierteste Strategie. Sie sucht eine |Kiste| k, die auf jeden Fall größer ist als |kNeu| und zusätzlich folgende Bedingungen erfüllt.
 Falls |k| bereits einen Nachfolger |links| hat, so muss |kNeu| größer als |links| sein.
 Falls |k| jedoch bereits zwei Nachfolger |links| und |rechts| hat, so muss einer der beiden folgenden Fälle erfüllt sein.
  Entweder der |links|  passt in |kNeu| und |k| hat Platz für |kNeu| und |rechts|,
          oder |rechts| passt in |kNeu| und |k| hat Platz für |kNeu| und |links|.
 Natürlich kann es auch sein, dass keiner der Fälle zutrifft. Trifft jedoch ein Fall zu, so packe die neue |Kiste| |kNeu| in den entsprechenden Zwischenraum.
\paragraph{FindeKleinereWurzel}
 Nun wurde noch eine Strategie implementiert, die eine Wurzel |kWurzel| sucht, die in |kNeu| passt.
 Diese ist recht einfach, sobald gefunden, wird |kWurzel| einfach in |kNeu| gepackt.
\paragraph{Standard-Strategien}
 Diese vier Strategien nenne ich auch \emph{Standard-Strategien}, wenn sie in der Reihenfolge FindeGrößerenLeeren, FindeZwischenraum, FindeHalbleeren, FindeKleinereWurzel verwendet werden.
\subsubsection{Offline-Algorithmus}
 Werden die Kisten vor dem inkrementellem Packen nach Volumen von groß nach klein sortiert,
  können die Strategien |FindeZwischenraum| und |FindeKleinereWurzel| ohne Beschränkung weggelassen werden.
 Diese suchen nämlich Kisten, die kleiner sind als die hinzuzufügende, welche jedoch wegen der Sortierung nicht existieren können.
 Da jedoch zur Sortierung alle Kisten bekannt sein müssen, handelt es sich nicht mehr um einen Online- sondern einen Offline-Algorithmus.
 Die Existenzberechtigung dieses Algorithmus' ergibt sich aus der Tatsache, dass bessere Ergebnisse bei vorheriger Sortierung erhalten werden können,
  als mit dem ursprünglichem Online-Algorithmus.
\clearpage

\section{Implementierung}
\lstset{basicstyle=\ttfamily}
 In diesem Abschnitt wird die Implementierung erläutert. Der Programmtext ist in Abschnitt \ref{code1} abgedruckt.

 Zunächst wurde ein Kern implementiert, welcher Kisten und Kistensätze sinnvoll abbildet und hilfreiche Funktionen zur Operation auf diesen bietet.
 Die Datentypen wurden als unveränderbare Objekte implementiert, um die Algorithmen zu vereinfachen.
\subsection{Kisten}
 Es gibt drei Arten von Kisten: |KisteLeer|, |KisteHalb| und |KisteVoll|.
 Eine |KisteLeer| enthält keine weitere |Kiste|, eine |KisteHalb| enthält eine |Kiste| und eine |KisteVoll| enthält zwei.
 Es wurde zunächst ein |trait Kiste| implementiert, welches eine externe Anwendungsschnittstelle bietet und eine interne Implementierungsschnittstelle,
 welche von den drei Unterklassen implementiert werden muss.
 Das |trait| wurde als |sealed| implementiert, das heißt, nur Klassen in der gleichen Datei dürfen dieses |trait| implementieren.
 Dies ermöglicht bessere Compiler-Unterstützung bei Pattern-matching, da bekannt ist, dass es nur genau 3 Unterklassen von Kiste gibt.

 Neben der Implementierung von Standardmethoden wie |hashCode: Int| zur Hashcode-Berechnung sowie |equals(Kiste): Kiste| als Definition der Äquivalenz
  wurden auch zahlreiche anwendungsspezifische Methoden und Felder implementiert.
 Hier beschränke ich mich auf die Vorstellung der wichtigsten Methoden und Felder.

 Eine |Kiste| besitzt die Felder |a,b,c: Int|, die jeweils das größte, zweitgrößte bzw.\ drittgrößte Außenmaß darstellen.
 Die Methode |+<(Kiste): Set[Kiste]| liefert als Ergebnis alle Möglichkeiten wie \emph{eine andere} |Kiste| in den durch \emph{diese} |Kiste| definierten Kistenbaum
  gepackt werden kann. Hierzu wird der gesamte Kistenbaum traversiert, indem die Methode |+<| rekursiv auf allen Nachfolgerkisten aufgerufen wird.
 Weiter wurden Methoden zum Vergleich zweier Kisten implementiert. Diese sind |>~(Kiste):Boolean| sowie |>=~(Kiste):Boolean|.
 Die erstere liefert zu einer anderen |Kiste der|, ob diese hineinpasst.
 Die zweite prüft indes, ob die |Kiste der| kleinere oder gleichgroße Maße hat, also ob |der.a <= a| und |der.b <= b| und |der.c <= c| gilt.
\subsection{Kistensatz}
 Da eine |Kiste| die Wurzel eines Kistenbaums ist und diesen repräsentiert, (betrachten Sie z.\ B.\ eine |KisteLeer| als einen einelementigen Kistenbaum),
  muss ein Kistensatz lediglich Referenzen auf die einzelnen |Kiste|-Objekte speichern.
 Es ergeben sich für die Datenstruktur, die den Kistensatz verwaltet, folgende drei Voraussetzungen. 
\begin{enumerate}
 \item Das Ersetzen eines [Teil]baums sollte möglichst billig sein.
  Da das Ersetzen einer |Kiste| als Löschen und anschließendes Hinzufügen dieser in den Baum implementiert ist,
   müssen sowohl die Lösch- als auch Hinzufügefunktionen kurze Laufzeiten haben.
  Da die Datenstruktur unveränderbar ist, liefert jede Veränderung in dem durch eine |Kiste| repräsentiertem Kistenbaum eine neues Objekt zurück.
  Dieses Objekt muss dann durch eine Lösch- und eine Hinzufügeoperation in den Baum des |Kistensatz| aktualisiert werden.
 \item Duplikate müssen zugelassen sein, da es passieren kann, das zwei Kisten genau die gleichen Maße haben.
 \item Die Kistenbäume eines Kistensatzes müssen so sortiert sein, dass ein Vergleichen nach Elementen zwischen zwei Kistensätzen billig ist.
\end{enumerate}
Diese drei Voraussetzungen erfüllt meiner Ansicht nach ein geordneter Binärbaum am besten.
Es wurde also die Scala-Standardklasse |TreeMap[Kiste,Int]| verwendet. Sie bildet jeweils eine |Kiste| $k$ auf eine Zahl $i_k > 1$ ab.
Diese Zahlen sind gleich der Anzahl der einzelnen |Kiste| (bzw. Kistenbaum) in diesem Kistensatz.
Somit erfolgt das Hinzufügen einer Kiste $k$ (Scala: |+(Kiste):Kistensatz|) mit dem Hinzufügen von $k \rightarrow 1$ in den Binärbaum,
  bzw. wenn $k$ schon erhalten ist, mit dem Setzen von $k \rightarrow i_k + 1$.
Analog erfolgt auch das Entfernen einer Kiste $k$ (Scala: |-(Kiste):Kistensatz|),
  daher wenn $k$ nicht enthalten oder $i_k = 1$ entferne $k$ aus dem Binärbaum, andernfalls setze $k \rightarrow i_k - 1$.

Außerdem wurde noch die Methode |+<(Kiste): Set[Kistensatz]| implementiert, die zu einer Kiste die Menge aller Kistensätze liefert,
in denen die neue Kiste in einer alten Kiste verpackt ist.
Weiter liefert |++<(Kiste): Set[Kistensatz]| zusätzlich zu den durch |+<| berechneten Kistensätzen den durch |+| berechneten |Kistensatz|.
Das heißt, |++<| liefert alle Kistensätze, die durch Hinzufügen der Kiste an einer beliebigen Stelle
 - also entweder in einer alten |Kist|e oder neben dem |Kistensatz| - möglich sind.
\subsection{Kistenpacker}
Da verschiedene Algorithmen entwickelt wurden, bietet es sich an, von mehreren Algorithmen verwendete Funktionen auszulagern.
Dies erfolgte in Scala durch Verwendung einer Klassenhierarchie.
Die Wurzel der Hierarchie ist das |trait| |Kistenpacker|. Dieses definiert das Feld |kisten: List[KisteLeer]| sowie die Methode |min: Kistensatz|.
|kisten| stellt die Liste der Kisten dar, die abgearbeitet werden sollen.
|kisten| muss jedoch nicht die Eingabeliste sein, es ist beispielsweise auch möglich, dass eine Unterklasse die Kisten sortiert bevor sie abgearbeitet werden.
Die Methode |min| soll den Kistensatz liefern, der möglichst geringes Außenvolumen aufweist.

Zusätzlich wurde die Klasse |SortierenderPacker| geschrieben,
  die das Feld |kisten| von |Kistenpacker| überschreibt und diese mit den Kisten der sortierten Eingabeliste |kistenListe| initialisiert.
\subsubsection{OptimalPacker}
Der Brute-force-Algorithmus wurde in dem |object OptimalPacker| implementiert.
Dieses wurde in etwa mit dem bereits im Entwurf dargestellten Code implementiert.
Die Methode |min|, die das Minimum zurück liefern soll, ruft |packe| auf, nachdem die Eingabeliste sortiert wurde.
Dann wird aus der so berechneten Menge aller Kistensätze das Minimum ausgewählt, wobei die Ordnung nach Volumen benutzt wird.
\subsubsection{AufteilenderPacker}
Der |Kistenpacker| |AufteilenderPacker| implementiert lediglich die Methode |min|.
In dieser wird zunächst der Wert |m| berechnet, der die Anzahl der Gruppen darstellt.
Dann wird das Feld |geteilteKisten| berechnet. Dieses soll die Gruppen als |Vector| von |List|en darstellen.
Beginnend bei einem |Vector| mit leeren |List|en wird ein |foldLeft| über die |kisten| durchgeführt.
Es werden die |Kiste|n in jedem Schritt an die im |Vector| nächste |List| angefügt.
Beim Erreichen des Endes vom |Vector| wird wieder vom Anfang des |Vector|s angefangen.
Anschließend wird für jede Gruppe das Minimum berechnet und anschließend durch einen |reduceLeft| Aufruf zu einem |Kistensatz| reduziert.
\subsubsection{OnlineAlgo und OfflinePacker}
Um den |OfflinePacker| gut implementieren zu können, wurde zunächst die Klasse |OnlineAlgo| implementiert, die den aktuellen |Kistensatz| und eine Liste von Strategien enthält.
Die Klasse |OnlineAlgo| implementiert zwei Methoden.
Die eine Methode |+(KisteLeer): OnlineAlgo| liefert den |OnlineAlgo| mit dem |Kistensatz|, der durch Anwendung der Strategien die neue |Kiste| enthält.
Hierbei werden alle Strategien mit |foldLeft| durchlaufen, wobei so lange versucht wird eine Strategie anzuwenden, bis eine passende gefunden wurde.
Wurde keine passende gefunden, wird die |Kiste| einfach neben den |Kistensatz| gestellt.
Die andere Methode |++(Seq[KisteLeer]): OnlineAlgo| liefert einfach zu einer Sequenz von Kisten durch nacheinander Aufrufen von |+| mit einem |foldLeft|-Aufruf einen
|OnlineAlgo| mit einem |Kistensatz| der alle neuen Kisten enthält.

Mit dieser Klasse gestaltet sich die Implementierung des |OfflinePacker|s sehr einfach.
Die Methode |min| muss lediglich die Methode |++| eines leeren |OnlineAlgo| aufrufen und dann den |Kistensatz|, der sich dadurch ergibt, zurückgeben.
\paragraph{Strategien}
Nun sollen noch die Implementierung der Strategien erläutert werden.
Diese folgt den Prinzipien, die in \ref{strats} beschrieben sind.
Es ist bei allen Strategien außer |FindeKleinereWurzel| jedoch noch nötig, den gesamten Kistenbaum ``über'' der gefunden Kiste anzupassen.
Da alle Datenstrukturen unveränderbar sind, muss entlang des Pfades nach oben der Kistenbaum neu aufgebaut werden.
Deswegen liefert die Methode |find| von |Kistensatz|, mit der die Kiste gesucht wird, einen Pfad bis zur neuen Kiste zurück.
Entlang diesem kann dann der Kistenbaum neu aufgebaut werden.

\clearpage

\section{Programmabläufe}
Nachfolgend sind mehrere Programmabläufe der verschiedenen Algorithmen abgebildet.
Soweit möglich, werden jeweils alle Algorithmen mit der gleichen Eingabeliste von Kisten ausgeführt.
\subsection{Kleines Beispiel}
Betrachten wir folgende 7 Kisten und deren Gesamtvolumen:
\begin{lstlisting}
scala> val kisten = ( 6 x  6 x  2) :: (6 x 4 x 2) :: (6 x 6 x 4) ::
                    ( 8 x  8 x  8) :: (8 x 8 x 6) :: (6 x 4 x 2) ::
                    (10 x 10 x 10) :: Nil
kisten: List[de.voodle.tim.bwinf.kisten.KisteLeer] =
  List(KisteLeer(6,6,2), KisteLeer(6,4,2), KisteLeer(6,6,4),
       KisteLeer(8,8,8), KisteLeer(8,8,6), KisteLeer(6,4,2),
       KisteLeer(10,10,10))

scala> val v = kisten.map(_.v).sum // Das Gesamtvolumen der Kisten
v: Int = 2208

\end{lstlisting}

\subsubsection*{OptimalPacker}
Als Erstes werden die Kisten mit |OptimalPacker| gepackt.
Es werden zunächst alle Möglichkeiten berechnet und anschließend nur das Minimum (vgl.\ Z.\ 31).

\begin{lstlisting}
// Alle Möglichkeiten erzeugen
scala> val optKss = OptimalPacker packe kisten
optKss:
 scala.collection.immutable.Set[de.voodle.tim.bwinf.kisten.Kistensatz]=
Set({
KisteHalb(8,8,8)
 \-KisteLeer(6,6,4),
KisteVoll(10,10,10)
 \-KisteVoll(8,8,6)
  \-KisteLeer(6,6,2)
  \-KisteLeer(6,4,2)
 \-KisteLeer(6,4,2)
}, {
KisteLeer(6,4,2),
KisteLeer(6,6,2),
KisteLeer(6,6,4),
KisteLeer(8,8,8),
KisteVoll(10,10,10)
 \-KisteLeer(8,8,6)
 \-KisteLeer(6,4,2)
}, {
KisteLeer(6,4,2),
KisteLeer(6,6,4),
KisteLeer(8,8,6),
KisteLeer(8,8,8),
KisteVoll(10,10,10)
 \-KisteLeer(6,6,2)
 \-KisteLeer(6,4,2)
}, { ...
scala> val anzahl = optKss.size // Anzahl berechneter Kistensätze
anzahl: Int = 216

scala> val opt = OptimalPacker min kisten
opt: de.voodle.tim.bwinf.kisten.Kistensatz = 
{
KisteVoll(8,8,6)
 \-KisteLeer(6,4,2)
 \-KisteLeer(6,4,2),
KisteHalb(10,10,10)
 \-KisteVoll(8,8,8)
  \-KisteLeer(6,6,4)
  \-KisteLeer(6,6,2)
}

scala> val v = opt.v
v: Int = 1384
\end{lstlisting}

\subsubsection*{AufteilenderPacker}
Als Zweites wird der Algorithmus |AufteilenderPacker| genutzt.
Der Aufteilungsgrad sei $t=4$.
Die Kisten werden also wie folgt in die vier Gruppen aufgeteilt.

\begin{tabular}{ll}
 1. und 5. Kiste & in die 1. Gruppe, \\
 2. und 6. Kiste & in die 2. Gruppe, \\
 3. und 7. Kiste & in die 3. Gruppe, \\
  sowie 4. Kiste & in die 4. Gruppe. \\
\end{tabular}

\begin{lstlisting}
scala> val auf = AufteilenderPacker(4) min kisten
auf: de.voodle.tim.bwinf.kisten.Kistensatz = 
{
KisteVoll(8,8,8)
 \-KisteLeer(6,4,2)
 \-KisteLeer(6,4,2),
KisteVoll(10,10,10)
 \-KisteHalb(8,8,6)
  \-KisteLeer(6,6,4)
 \-KisteLeer(6,6,2)
}

scala> val v = auf.v
v: Int = 1512
\end{lstlisting}
 Wir sehen hier, dass das Volumen größer ist als beim Optimum.
 Dies liegt daran, dass in diesem Beispiel die Kisten |(10 x 10 x 10)| und |(8 x 8 x 6)| in der gleichen Kistengruppen sind.
 Diese müssten jedoch beide Wurzel sein, um eine optimale Packung erreichen zu können.

\subsubsection*{OfflinePacker}
 Schließlich wird mit |OfflinePacker| gepackt.
 Es wird ein |OfflinePacker| mit leerer Parameterliste erzeugt, somit werden die |standardStrategien| benutzt und die Kisten vor dem Packen sortiert.
\begin{lstlisting}
scala> val off = OfflinePacker() min kisten
off: de.voodle.tim.bwinf.kisten.Kistensatz = 
{
KisteLeer(6,4,2),
KisteHalb(8,8,6)
 \-KisteLeer(6,6,4),
KisteHalb(10,10,10)
 \-KisteVoll(8,8,8)
  \-KisteLeer(6,6,2)
  \-KisteLeer(6,4,2)
}

scala> off.v
res2: Int = 1432

\end{lstlisting}
 Hier ist das Volumen ebenfalls größer als beim Optimum |opt|, jedoch kleiner als |auf|.
 Die Strategien des |OfflinePacker| finden für die letzte zu packende Kiste (6 x 4 x 2) keinen Platz mehr.

\subsection{Algorithmus-Vergleich}
\label{algcont}
 In diesem Abschnitt sollen nun die Algorithmen für größere Beispielen miteinander verglichen werden.
 Wir betrachten insbesondere die Packdichte und die Laufzeit.
 Wir vergleichen bei den kleinen Beispielen die Algorithmen Brute-force, Brute-force nach Aufteilung, sowie OfflinePacker.
 Bei den mittleren und größeren Beispielen müssen wir auf den Algorithmus Brute-force verzichten, denn dieser lässt sich nicht mehr in passabler Zeit ausführen.
\paragraph{Hinweis}
 Alle Zeitangaben sind in Millisekunden.
 Das Verhältnis zwischen der Summe aller Volumina der ungepackten Kisten zu der Summe der Endvolumina der äußeren Kisten bezeichne ich als \emph{Komprimierungsgrad}.
 Es gilt also
 \[ \text{Komprimierungsgrad} = \dfrac{\sum\text{Volumina der ungepackten Kisten}}{\sum\text{Envolumina der äußeren Kisten}} .\]
\paragraph{Hilfsfunktion}
 Zunächst wird eine Hilfsfunktion definiert, die zu einer Anzahl eine Funktion zurück liefert, die entsprechend viele Kisten zufällig erzeugt.
 Die Kisten haben dabei maximal eine Abmessung von 80 x 80 x 80.
\begin{lstlisting}
scala> def kisten(anzahl: Int) =
  () => StandardKisten.zufallKisten(80,80,80) take anzahl toList
kisten: (anzahl: Int)() => List[de.voodle.tim.bwinf.kisten.KisteLeer]
\end{lstlisting}
 Außerdem definieren wir uns noch eine Funktion |großeKisten|, die ähnlich wie die obere Funktion zufällig erzeugte Kisten zurück gibt.
 Allerdings erzeugt diese Funktion Kisten mit Abmessungen bis zu 1000 x 1000 x 1000.
\begin{lstlisting}
scala> def großeKisten(anzahl: Int) =
  () => StandardKisten.zufallKisten(1000,1000,1000) take anzahl toList
großeKisten:(anzahl: Int)()=>List[de.voodle.tim.bwinf.kisten.KisteLeer]
\end{lstlisting}

\subsection{Kleine Anzahl Kisten}
\label{optAlg}
 Unter einer kleinen Anzahl Kisten verstehe ich etwa 10 Kisten. Nachfolgend wird also immer mit 8, 10 bzw. 12 Kisten gemessen.
 Es werden nun nacheinander verschieden Algorithmen - bzw.\ Varianten davon - auf Packdichte und Laufzeit analysiert.
 Nachfolgend wird jeweils für jeden Algorithmus eine Musterausführung dargestellt.
 Anschließend werden die gewonnenen Ergebnisse in einer Tabelle zusammengefasst gegenübergestellt.
\paragraph{OptimalPacker}
 Der Algorithmus |OptimalPacker| wird 1000-mal nacheinander mit 8, 10, bzw.\ 12 Kisten ausgeführt.
 Aufgrund der schlechten Laufzeitkomplexität, werden bei 12 Kisten jedoch nur 100 Abläufe ausgeführt.
 Die folgenden Zeilen 1-7 zeigen das Beispiel für 8 Kisten.
\begin{lstlisting}
scala> val (optKd, optZeit) =
  Utils.bench(OptimalPacker, kisten(8), 1000, verbose = false)
Packer: OptimalPacker
Wiederholungen  Komprimierungsgrad      Zeit
8       	1,4                     3
optKd: BigDecimal = 1.4043923926972448
optZeit: Long = 3

[...] // Aufrufe mit 10 bzw. 12 Kisten
\end{lstlisting}
\paragraph{AufteilenderPacker}
Den Aufteilungsgrad $t$ setze ich zunächst immer auf die Hälfte der Anzahl von Kisten, also 4, 6 bzw. 8.
Anschließend teste ich nochmal mit Kistenanzahlen von 10 und 12, jedoch mit gleichbleibendem Aufteilungsgrad $t = 4$.
Gezeigt ist hier ebenfalls das Beispiel für 8 Kisten.
\begin{lstlisting}
scala> val (aufKd, aufZeit) =
  Utils.bench(AufteilenderPacker(4), kisten(8), 1000, verbose = false)
Packer: AufteilenderPacker(4)
Wiederholungen  Komprimierungsgrad      Zeit
1000            1,23                    0
aufKd: BigDecimal = 1.228
aufZeit: Long = 0

[...] // Aufrufe mit 10 bzw. 12 Kisten
\end{lstlisting}

\paragraph{OfflinePacker}
Schließlich wird auf gleiche Weise der |OfflinePacker| mit den Standard-Strategien getestet. Gezeigt ist wieder der Fall für 8 Kisten.
\begin{lstlisting}
scala> val (offKd, offZeit) =
  Utils.bench(OfflinePacker(), kisten(8), 1000, verbose = false)
Packer: OfflinePacker(OnlineAlgo({

},List(FindeGrößerenLeeren, FindeZwischenraum,
       FindeHalbleeren, FindeKleinereWurzel)),true)
Wiederholungen  Komprimierungsgrad      Zeit
1000            1,32                    0
offKd: BigDecimal = 1.325
offZeit: Long = 0

[...] // Aufrufe mit 10 bzw. 12 Kisten
\end{lstlisting}
In den beiden folgenden Tabellen sind alle Ergebnisse der Beispielprogrammausführungen zusammengefasst.
\subsubsection*{Komprimierungsgrad}
\begin{center}
\begin{tabular}{lrrr}
\textbf{Algorithmus} 	& \textbf{8 Kisten} 	& \textbf{10 Kisten} 	& \textbf{12 Kisten} \vspace{3.6pt}\\
 OptimalPacker       	& 1,40               	& 1,46              	& 1,54               	\\
 AufteilenderPacker ($t=\frac{n}{2}$)  	& 1,23	& 1,27			& 1,32			\\
 AufteilenderPacker ($t=4$)	 	& 1,22	& 1,19			& 1,22			\\
 OfflinePacker		& 1,32			& 1,39			& 1,43			\\
\end{tabular}
\end{center}
 Den höchsten Komprimierungsgrad erzielt wie erwartet der |OptimalPacker|.
 Der |AufteilenderPacker| fällt deutlich hinter den |OfflinePacker| zurück.
 Außerdem ist der Komprimierungsgrad bei $t=\frac{n}{2}$ ein Stück weit größer als bei $t=4$.
 Auffallend ist außerdem, dass bei 8 Kisten beide Varianten des Algorithmus' |AufteilenderPacker| genau die gleichen Parameter übergeben bekommen (denn $t=\frac{8}{2}=4$)
  und trotzdem abweichende Komprimierungsgrade erreichen. Dies liegt daran, dass die Kisten für jeden Aufruf neu zufällig erzeugt werden und somit kleine Abweichungen unvermeidlich sind.

\subsubsection*{Laufzeit [in Millisekunden]}
\begin{center}
\begin{tabular}{lrrr}
\vspace{3.6pt}
\textbf{Algorithmus} 	& \textbf{8 Kisten} 	& \textbf{10 Kisten} 	& \textbf{12 Kisten} \\
 OptimalPacker       	& 3               	& 49              	& 1273                \\
 AufteilenderPacker ($t=\frac{n}{2}$)  	& 0	& 0			& 0		       \\
 AufteilenderPacker ($t=4$)	 	& 0	& 0			& 0			\\
 OfflinePacker		& 0			& 0			& 0			 \\
\end{tabular}
\end{center}
 Bereits hier kann erkannt werden, wie schlecht der Brute-force-Algorithmus skaliert.
 Bei lediglich 2 Kisten mehr braucht |OptimalPacker| bereits bis zu 25 mal ($49 * 25 = 1225 < 1273$) länger.

 Es lässt sich ebenfalls erkennen, dass sowohl |OfflinePacker| als auch |AufteilenderPacker| - gegenüber dem |OptimalPacker| -
  bei kleinen Kistenzahlen wesentlich schneller sind, so dass die Laufzeiten (bei diesen kleinen Kistenanzahlen) nicht einmal gemessen werden können
  und somit zu 0 gerundet werden.

\subsection{Mittlere Anzahl Kisten}
 Nach dem Ausscheiden von |OptimalPacker| testen wir nun in der Größenordnung von 100 Kisten.
 Genauer gesagt, testen wir mit den Anzahlen 88, 100 sowie 128.
 Allerdings testen wir nun nicht mehr mit 1000 Wiederholungen, sondern mit 256 Wiederholungen, um zügig testen zu können.
 Um zu sehen, wie sich die Algorithmen bei unterschiedlichen Größen der Kisten verhalten, wird zusätzlich mit der Funktion |großeKisten| getestet.
 Beim aufteilenden Packer ist natürlich wieder die Frage, welcher Teilungsfaktor sinnvoll ist.
 Er sollte auf jeden Fall nicht größer als 10 sein, da die Laufzeit exponentiell mit $t$ steigt und größere Werte hier leider nicht praktikabel sind.
 Ich entschied mich für 8 und 10 als Teilungsfaktor.
\paragraph{}
 In den beiden folgenden Tabellen sind alle Ergebnisse der Beispielprogrammausführungen zusammengefasst.

\subsubsection*{Komprimierungsgrad}
\begin{center}
\begin{tabular}{lrrr}
\vspace*{.42em}
\textbf{Algorithmus} 	& \textbf{88 Kisten} 	& \textbf{100 Kisten} 	& \textbf{128 Kisten} 	\\
\textit{Kleine Kisten} & & & \\
 AufteilenderPacker ($t=8$)  	& 1,40		& 1,39			& 1,40			\\
 AufteilenderPacker ($t=10$)	& 1,47		& 1,47			& 1,47			\\
 OfflinePacker			& 1,96		& 2,00			& 2,07			\\
\vspace*{.42em}
 OfflinePacker (eine Strategie)	& 1,24		& 1,25			& 1,25			\\
\textit{Große Kisten} & & & \\
 AufteilenderPacker ($t=8$)  	& 1,41		& 1,40			& 1,41			\\
 AufteilenderPacker ($t=10$)	& 1,46		& 1,46			& 1,44			\\
 OfflinePacker			& 2,49		& 2,57			& 2,72			\\
 OfflinePacker (eine Strategie)	& 2,27		& 2,35			& 2,48			\\
\end{tabular}
\end{center}
 Der erreichte Komprimierungsgrad mit |OfflinePacker| ist größer als 2 und damit deutlich über dem Komprimierungsgrad, der mit |AufteilenderPacker|
  erreicht werden konnte. Außerdem lässt sich beobachten, dass der Komprimierungsgrad mit steigender Kistenzahl steigt.
 Dies lässt sich damit erklären, dass die Kistensätze durch mehr Kisten, besser ``gesättigt'' werden können. 
 Interessanterweise liefert der aufteilende Packer auch mit $t=10$ kaum bessere Werte als mit $t=8$.
 Allerdings braucht er - wie wir unten noch genauer betrachten - deutlich länger, da die Laufzeit exponentiell mit $t$ steigt.
 Als Vergleich werden hier noch die Ergebnisse des einfachsten |OfflinePacker|s mit der einzigen Strategie |FindeGrößerenLeeren| und ohne Sortieren vor dem Packen angegeben.

\subsubsection*{Laufzeit [in Millisekunden]}
\begin{center}
\begin{tabular}{lrrr}
\vspace*{.42em}
\textbf{Algorithmus} 	& \textbf{88 Kisten} 	& \textbf{100 Kisten} 	& \textbf{128 Kisten} 	\\
\textit{Kleine Kisten} & & & \\
 AufteilenderPacker ($t=8$)  	& 33		& 30			& 49			\\
 AufteilenderPacker ($t=10$)  	& 375		& 490			& 792			\\
 OfflinePacker			& 5		& 7			& 13			\\
\vspace*{.42em}
 OfflinePacker (eine Strategie)	& 3		& 4			& 7			\\
\textit{Große Kisten} & & & \\
 AufteilenderPacker ($t=8$)  	& 41		& 37			& 59			\\
 AufteilenderPacker ($t=10$)	& 477		& 680			& 732			\\
 OfflinePacker			& 3		& 4			& 7			\\
 OfflinePacker (eine Strategie)	& 2		& 3			& 5			\\
\end{tabular}
\end{center}
 Über die Laufzeit lassen sich hier nur Vermutungen anstellen, da die Werte absolut gesehen klein bleiben.
 Beim Algorithmus |AufteilenderPacker| ist der sprunghafte Anstieg der Laufzeit bei leicht steigendem $t$ zu beobachten.
 Dies kommt durch die exponentielle Laufzeit des |OptimalPacker|, welcher vom aufteilenden Packer benutzt wird.
 Absolut gesehen lässt sich noch die niedrigere Laufzeit des |OfflinePacker| erkennen.
 Als vernünftige Indizen für die Laufzeitkomplexität sind die Ergebnisse nicht geeignet,
  da die Abweichungen bei diesen kleineren Beispielen relativ gesehen noch zu hoch sind.

\subsection{Große Anzahl Kisten}
 Nun sollen große Anzahlen von Kisten getestet werden. Genauer gesagt 800, 1600 und 3200 Kisten.
 Neben dem Variieren der Kistenanzahl, werden wir außerdem noch die Größe der Kisten verändern.
 Um die absolute Laufzeit niedrig zu halten, werden außerdem nur noch 64 Wiederholungen ausgeführt.
 Außerdem teste ich weiterhin mit den Teilungsfaktoren 8 bzw.\ 10.

\subsubsection*{Komprimierungsgrad}
\begin{center}
\begin{tabular}{lrrr}
\vspace*{.42em}
\textbf{Algorithmus} 	& \textbf{800 Kisten} 	& \textbf{1600 Kisten} 	& \textbf{3200 Kisten} \\
\textit{Kleine Kisten} & & & \\
 AufteilenderPacker ($t=8$)  	& 1,40		& 1,40			& 1,40			\\
 AufteilenderPacker ($t=10$)  	& 1,46		& 1,47			& 1,47			\\
 OfflinePacker			& 2,80		& 3,12			& 3,47			\\
 \vspace{.42em}
 OfflinePacker (eine Strategie)	& 1,28		& 1,29			& 1,30			\\
\textit{Große Kisten} & & & \\
 AufteilenderPacker ($t=8$)  	& 1,41		& 1,41			& 1,41			\\
 AufteilenderPacker ($t=10$)  	& 1,46		& 1,47			& 1,47			\\
 OfflinePacker			& 4,45		& 5,49			& 6,64			\\
 OfflinePacker (eine Strategie)	& 4,01		& 4,80			& 5,80			\\
\end{tabular}
\end{center}
 Der mit |OfflinePacker| erreichte Komprimierungsgrad ist jetzt sogar deutlich über doppelt so groß wie der Komprimierungsgrad des aufteilenden Packers.
 Dies lässt sich im direktem Vergleich erkennen.
 Während der Komprimierungsgrad bei |AufteilenderPacker| konstant schwach auf etwa 1,4 bleibt,
  steigt der Komprimierungsgrad beim |OfflinePacker| bis auf über 6 an.
 Dies lässt sich an der Funktionsweise erklären.
 Da der |OfflinePacker| immer im gesamtem Kistensatz nach einem neuem Platz für eine Kiste sucht, kann er die Anzahl der Kisten ausnutzen.
 Der aufteilende Packer packt jedoch immer $t$, also 8 bzw. 10 Kisten ineinander.
 Diese Kistengruppen haben jeweils etwa die gleiche Dichte, also auch der gesamte Kistensatz.

\subsubsection*{Laufzeit [in Millisekunden]}
\begin{center}
\begin{tabular}{lrrr}
\vspace*{.42em}
\textbf{Algorithmus} 	& \textbf{800 Kisten} 	& \textbf{1.600 Kisten} & \textbf{3.200 Kisten} \\
\textit{Kleine Kisten} & & & \\
 AufteilenderPacker ($t=8$)	& 346		& 670			& 1.492			\\
 AufteilenderPacker ($t=10$)  	& 4.074		& 8.352			& 18.198			\\
 OfflinePacker			& 348		& 1.340			& 5.290			\\
 OfflinePacker (eine Strategie)	& 373		& 2.237			& 7.056			\\
\vspace*{.42em}
\textit{Große Kisten} & & & \\
 AufteilenderPacker ($t=8$)  	& 413		& 837			& 1.719			\\
 AufteilenderPacker ($t=10$)  	& 4.074		& 8.352			& 18.198			\\
 OfflinePacker			& 182		& 633			& 2.302			\\
 OfflinePacker (eine Strategie)	& 180		& 669			& 2.527			\\
\end{tabular}
\end{center}
 Bei dem aufteilenden Packer lässt sich ganz klar die lineare Laufzeitkomplexität beobachten.
 Der Laufzeitsanstieg von 346 ms auf 670 ms ($346 \cdot 2 = 692$) respektive 1492 ms ($670 \cdot 2 = 1340$) bei jeweils verdoppelter Anzahl Kisten ist hierfür ein guter Beleg.

 Außerdem lässt sich bereits - trotz absolut schnellerer Laufzeit - die schlechtere Komplexitätsstufe von |OfflinePacker| erkennen.
 Denn wie erkennbar, steigt die Laufzeit nicht linear mit der Kistenanzahl, sondern quadratisch.
 Der Algorithmus braucht 348 ms bei 800 kleinen Kisten, $1340\ (\approx 348 \cdot 4 = 1392)$ ms bei 1.600 Kisten sowie $5290\ (\approx 1340 \cdot 4 = 5360)$ ms bei 3.200 Kisten.
 Bei Verdopplung der Kisten braucht der |OfflinePacker| also etwa viermal so lange.

\subsection{Riesige Anzahl Kisten}
 Abschließend vergleichen wir noch kurz das Laufzeitverhalten bei 10.000 respektive 80.000 Kisten.
 Ich entschied mich für den Teilungsgrad $t=9$, da mit $t=10$ kaum bessere Ergebnisse erzielt werden können, aber eine deutlich höhere Laufzeit erfordert wird.
 Wir testen jeweils nur noch mit zwei Wiederholungen.

\subsubsection*{Komprimierungsgrad}
\begin{center}
\begin{tabular}{lrrr}
\vspace*{.42em}
\textbf{Algorithmus}   & \textbf{10.000 Kisten} & \textbf{80.000 Kisten}\\
\textit{Kleine Kisten} & & \\
 AufteilenderPacker ($t=9$)  	& 1,43		& 1,43			\\
\vspace*{.42em}
 OfflinePacker			& 4,02		& 9,07			\\
\textit{Große Kisten} & & \\
 AufteilenderPacker ($t=9$)  	& 1,44		& 1,44			\\
 OfflinePacker			& 9,17		& 16,89			\\
\end{tabular}
\end{center}
 Bei dem |OfflinePacker| lassen sich sogar Komprimierungsgrade bis deutlich über 10 beobachten.
 Das heißt, bei 80.000 Kisten, deren Größen zufällig von 1 x 1 x 1 bis 1000 x 1000 x 1000 verteilt sind,
  lässt sich der Platzverbrauch auf weniger als ein Zehntel reduzieren.
 Auch bei kleineren Kisten, ist bei über 10.000 Kisten mit etwa einem Viertel des Anfangsvolumen als Endvolumen zu rechnen.

 Der aufteilende |Kistenpacker| erreicht wie erwartet weiterhin einen Komprimierungsgrad von ca. 1,4.
 Dieser Komprimierungsgrad lässt sich in der Regel mit jeder etwas größeren Kistenanzahl erreichen.
 
\subsubsection*{Laufzeit [in Millisekunden]}
\begin{center}
\begin{tabular}{lrrr}
\vspace*{.42em}
\textbf{Algorithmus} 	& \textbf{10.000 Kisten} & \textbf{80.000 Kisten} \\
\textit{Kleine Kisten} & & \\
 AufteilenderPacker ($t=9$)  	& 14.620	& 113.671		\\
 OfflinePacker			& 56.835	& 2.024.376		\\
\vspace*{.42em}
\textit{Große Kisten} & & \\
 AufteilenderPacker ($t=9$)  	& 17.142	& 146.574		\\
 OfflinePacker			& 18.898	& 934.460		\\
\end{tabular}
\end{center}

 Die Laufzeit des |OfflinePacker| ist nur noch gerade so beherrschbar. 2.024.376 Millisekunden sind eine knappe Stunde.
 Hier zeigt sich ganz klar die bessere Laufzeitkomplexität des aufteilenden Packers.
 Die Linearität der Laufzeit lässt noch deutlich höhere Kistenanzahlen zu, wie wir später nochmal sehen werden.

\subsection{Fazit}
Zusammenfassend lässt sich sagen, dass der |OptimalPacker| für Kistenanzahlen bis 12 der beste ist, denn er liefert garantiert optimale Ergebnisse.
Ab der 13. Kiste ist seine Laufzeit jedoch nicht mehr kontrollierbar.
Für Kistenzahlen von 13 bis 3.200 empfiehlt sich, den |OfflinePacker| mit Standard-Strategien zu benutzen.
Er bietet hier gute Komprimierungsgrade und trotzdem akzeptable Laufzeit.
Der Algorithmus |AufteilendePacker| lässt sich nur sinnvoll ab 3.200 Kisten einsetzen, da er deutlich schlechtere Komprimierungsgrade als der |OfflinePacker| liefert.
Er hat jedoch den Vorteil der linearen Laufzeitkomplexität und somit kann er auch größere Kistenanzahlen in praktikabler Zeit bewältigen.
Um den Komprimierungsgrad zu erhöhen, müsste jedoch der Teilungsfaktor erhöht werden, was wiederum die Laufzeit dramatisch wachsen lassen würde.
Bemerken muss man hier jedoch auch, dass sich der aufteilende Packer auch über mehrere Prozessoren skalieren lässt.
Dies ermöglicht Berechnungen mit bis zu über hunderttausend Kisten.

Zuletzt soll noch einmal demonstriert werden, was die lineare Komplexität in der Praxis heißt.
Nachfolgend ein Benchmark des Algorithmus' |AufteilenderPacker| mit $t=8$ und 888.888 Kisten.
\begin{lstlisting}
scala> Utils.bench(AufteilenderPacker(8), kisten(888888), 1, false)
Packer: AufteilenderPacker(8)
Wiederholungen  Komprimierungsgrad      Zeit
1               1,4                     1716431
res5: (BigDecimal, Long) = (1.401,1716431)
\end{lstlisting}
Wie zu sehen, wurden 1.716.431 Millisekunden, also etwa 1.716 Sekunden benutzt. Dies ist eine knappe halbe Stunde und somit noch völlig im machbaren Bereich.
\clearpage
\section{Programmnutzung}
\lstset{basicstyle=\ttfamily}
Die Nutzung des Programms erfolgt primär über eine Scala-Console mit richtig eingestelltem Classpath.
Um dies einfach zu erreichen, empfehle ich, die Scala-Console nach Anleitung im Kapitel ``Allgemeines'' (\ref{allgemein}. \ref{start}) zu starten.
Dann haben Sie auch bereits alle nötigen Module, Klassen, etc.\ importiert.

\subsection{Erzeugen der Kisten}
Kisten können manuell eingegeben werden, aber auch zufällig erzeugt werden.
Zur manuelle Eingabe gibt man die 3 Maße in Klammern mit |x| getrennt an.
Dafür geben Sie beispielsweise folgendes ein.
\begin{lstlisting}
scala>val kiste = (8 x 8 x 4)
kiste: de.voodle.tim.bwinf.kisten.KisteLeer = KisteLeer(8,8,4)

\end{lstlisting}
Um zufällige Kisten zu erzeugen, bedient man sich der Methode |zufallKisten| aus |StandardKisten|.
Hierbei müssen die Maximalwerte der Maße angegeben werden.
Die Methode erzeugt eine unendliche Liste, also muss noch angeben werden, wie viele Kisten erzeugt werden sollen.
Danach sollte dies mit |toList| in eine normale Scala-Liste umgewandelt werden.
Um beispielsweise 800 Kisten mit Maximalwerten von 80 in jeder Dimension zu erzeugen, geht man wie folgt vor.
\begin{lstlisting}
scala> import StandardKisten._
import StandardKisten._

scala> val kisten = zufallKisten(80,80,80) take 800 toList
kisten: List[de.voodle.tim.bwinf.kisten.KisteLeer] =
  List(KisteLeer(59,26,6),  KisteLeer(79,33,32), KisteLeer(78,34,25),
       KisteLeer(79,13,1),  KisteLeer(67,21,7),  KisteLeer(54,2,1),
       KisteLeer(49,41,13), KisteLeer(54,23,11), KisteLeer(57,53,12),
       KisteLeer(61,37,37), KisteLeer(70,64,4),  KisteLeer(77,28,3),
       KisteLeer(73,67,34), KisteLeer(71,55,28), KisteLeer(44,25,9),
       KisteLeer(52,28,19), KisteLeer(31,22,10), KisteLeer(54,23,16),
       KisteLeer(50,33,15), KisteLeer(69,15,6),  KisteLeer(72,71,53),
       KisteLeer(53,50,33), KisteLeer(78,67,32), KisteLeer(74,21,1),
       KisteLeer(60,10,3),  KisteLeer(59,28,13), KisteLeer(58,44,12),
       KisteLeer(48,41,5),  KisteLeer(75,70,63), KisteLeer(59,46,43),
       KisteLeer(51,32,31), KisteLeer(44,37,29), KisteLeer(77,57,30),
       KisteLeer(54,50,32), KisteLeer(59,45,21), KisteLeer(40,21,...
\end{lstlisting}
Es wurden außerdem noch einige gängigen Kistenmaße aus dem Internet\footnote{\href{http://kartonfritze.de}{http://kartonfritze.de}
                                                                        sowie \href{http://umzugskarton.de}{http://umzugskarton.de}} recherchiert.
Diese finden Sie ebenfalls unter |StandardKisten|.
Nachfolgend zeige ich einige Beispiele.
\begin{lstlisting}
scala> val kisten =
         dinodrei :: buecherkarton :: umzugskarton :: paroli :: Nil
kisten: List[de.voodle.tim.bwinf.kisten.KisteLeer] =
  List(KisteLeer(43,35,10), KisteLeer(41,34,32), KisteLeer(63,33,31),
       KisteLeer(61,35,34))

\end{lstlisting}
\subsection{Erstellung der Schachtelung}
Die Schachtelung erstellt man, indem die Methode |min| eines |Kistenpacker|s aufgerufen wird.
Um die Schachtelung mit |OptimalPacker| auszuführen, gibt man Folgendes in die Konsole ein.
\label{demo_kisten}
\begin{lstlisting}
scala> val kisten =
         zufallKisten(80,80,80) take 8 toList // Kisten erzeugen
kisten: List[de.voodle.tim.bwinf.kisten.KisteLeer] =
  List(KisteLeer(72,69,21), KisteLeer(67,38,28), KisteLeer(66,57,43),
       KisteLeer(64,42,40), KisteLeer(58,46,25), KisteLeer(70,53,37),
       KisteLeer(28,15,1),  KisteLeer(45,12,8))

scala> val min = OptimalPacker min kisten
min: de.voodle.tim.bwinf.kisten.Kistensatz = 
{
KisteHalb(58,46,25)
 \-KisteLeer(45,12,8),
KisteVoll(66,57,43)
 \-KisteLeer(64,42,40)
 \-KisteLeer(28,15,1),
KisteHalb(70,53,37)
 \-KisteLeer(67,38,28),
KisteLeer(72,69,21)
}
\end{lstlisting}
Um mit dem aufteilendem |Kistenpacker| Kistenschachtelungen zu erstellen, muss noch der Teilungsfaktor angegeben werden.
\begin{lstlisting}
scala> val min = AufteilenderPacker(5) min kisten // Teilungsfaktor t=5
min: de.voodle.tim.bwinf.kisten.Kistensatz = 
{
KisteHalb(64,42,40)
 \-KisteLeer(45,12,8),
KisteVoll(66,57,43)
 \-KisteLeer(58,46,25)
 \-KisteLeer(28,15,1),
KisteHalb(70,53,37)
 \-KisteLeer(67,38,28),
KisteLeer(72,69,21)
}
\end{lstlisting}
Für den |OfflinePacker| können Strategien angegeben werden. Bei Eingabe einer leeren Parameterliste werden einfach die Standard-Strategien benutzt.
Außerdem kann angegeben werden, ob sortiert werden soll oder nicht.
In dem folgendem Beispiel werden alle von mir implementierten Strategien benutzt, sowie wird die Sortierung nicht benutzt.
Es kann jedoch auch die Reihenfolge variiert werden.
\begin{lstlisting}
scala> val strategien = FindeHalbleeren   :: FindeGrößerenLeeren ::
                        FindeZwischenraum :: FindeKleinereWurzel :: Nil
strategien: List[de.voodle.tim.bwinf.kisten.Strategie] =
  List(FindeHalbleeren, FindeGrößerenLeeren,
       FindeZwischenraum, FindeKleinereWurzel)

scala> val min =OfflinePacker(strategien, sortieren = false) min kisten
min: de.voodle.tim.bwinf.kisten.Kistensatz = 
{
KisteVoll(66,57,43)
 \-KisteLeer(64,42,40)
 \-KisteLeer(28,15,1),
KisteLeer(67,38,28),
KisteVoll(70,53,37)
 \-KisteLeer(58,46,25)
 \-KisteLeer(45,12,8),
KisteLeer(72,69,21)
}

\end{lstlisting}
\subsection{Benchmarking eines Algorithmus'}
\subsubsection{Volumina}
 Hier wird beschrieben, wie Volumina von Kistensätzen und Listen von Kisten berechnet werden können.
 Dies macht insbesondere Sinn, um den Komprimierungsgrad, also Verhältnis von Anfangsvolumen und Endvolumen zu berechnen.

\paragraph{Volumen einer Kistenliste}
 Angenommen, wir haben in |kisten| eine |List| von |Kiste|n, dann kann das Volumen wie folgt berechnet werden.
 Die |kisten| sind hierbei die oben in \ref{demo_kisten} erzeugten.
\begin{lstlisting}
scala> val v = kisten.map(_.v).sum
v: Int = 653612
\end{lstlisting}
 Genauso kann mit einem |Set|, also einer Menge von Kisten verfahren werden.
\paragraph{Volumen eines Kistensatzes}
 Wenn |ks| ein Kistensatz ist, dann lässt sich das Volumen direkt abfragen.
\begin{lstlisting}
scala> val kv = ks.v
kv: Int = 286593
\end{lstlisting}
\paragraph{Komprimierungsgrad}
 Nun haben wir das Startvolumen $v$ berechnet, sowie das Endvolumen $kv$.
 Die Berechnung des Komprimierungsgrad ist schließlich nur noch eine Division.
 Allerdings empfiehlt sich |Double| zu benutzen, um Fließkommazahlen zu erhalten.
\begin{lstlisting}
scala> val komprimierungsgrad = v.toDouble / kv
komprimierungsgrad: Double = 2,2806279
\end{lstlisting}

\subsubsection{Benchmarking}
 Um möglichst gute Vergleiche zwischen Algorithmen liefern zu können,
  wurde zum Benchmarking die Methode |def bench(Kistenpacker, () => Seq[KisteLeer], Int, Boolean): (BigDecimal, Long)| im Modul |Utils| implementiert.
 Diese gibt zu einem |Kistenpacker|,
  einer Funktion zur Erzeugung von Kisten sowie einer Anzahl von Wiederholungen den durchschnittlich erreichten Komprimierungsgrad sowie die benötigte Durchschnittszeit in Millisekunden zurück.
 Zusätzlich kann die Ausgabe beschränkt werden, indem zusätzlich |false| am Ende der Parameterliste übergeben wird.
 Beispielsweise kann so ein |OfflinePacker| nur mit der Strategie |FindeHalbleeren| der 80 Kisten mit zufälliger Größe packt, wie folgt getestet werden.
\begin{lstlisting}
scala> val (packungsdichte, zeit) =
  Utils.bench(OfflinePacker(FindeHalbleeren :: Nil),
              () => zufallKisten(80,80,80) take 80 toList, 2) 
Packer: OfflinePacker(OnlineAlgo({

},List(FindeHalbleeren)),true)
Minimum: {
KisteLeer(18,15,5),
[..] // Kisten ausgelassen
KisteLeer(79,50,38)
}
Minimum: {
KisteLeer(18,12,6),
[..] // Kisten ausgelassen
KisteLeer(79,72,38)
}
Wiederholungen  Komprimierungsgrad      Zeit
2               1                       5
packungsdichte: BigDecimal = 1
zeit: Long = 5
\end{lstlisting}
 Nur die Strategie |FindeHalbleeren| zu benutzen, ist natürlich nicht sinnvoll,
  denn ohne zusätzliche Strategie wird nie eine halbleere Kiste erzeugt. Der Komprimierungsgrad ist und bleibt einfach 1.
 Hier dient das Beispiel lediglich zur Demonstration der Benutzung.

\addtolength{\textheight}{0cm}
\lstset{xleftmargin=-.8cm,xrightmargin=-.8cm}
\fontsize{8pt}{8.4pt}
\lstset{basicstyle=\ttfamily\small}
\clearpage
\section{Programmtext}
\label{code1}
Alle Quelldateien dieser Aufgabe finden sich auf der CD unter |Aufgabe1/src/|.
\subsection{Kisten}
\begin{lstlisting}
package de.voodle.tim.bwinf.kisten
import math._

sealed abstract class Kiste extends Ordered[Kiste] {
  val a,b,c: Int
  final val v = a*b*c // cache it!

  final def +<(der: Kiste) =
    if(this >~ der)   // Passt der hinein?-
      this +<< der  // Dann packe ihn ein!
    else Set.empty // Sonst lass' es sein.

  protected def +<<(der:Kiste): Set[Kiste]

  // Folgende Methoden prüfen unabhänig vom Inhalt!
  final def >~(der: Kiste) = a > der.a && b > der.b && c > der.c
  final def >=~(der: Kiste) = a >= der.a && b >= der.b && c >= der.c

  // Benutzung intern.
  protected[kisten] def <(x: Int, y: Int, z: Int) =
    a < x && b < y && c < z
  protected[kisten] def tuple_/:[T](start: T)(f: (T, Int,Int,Int) => T) =
    f(f(f(f(f(f(start, a,b,c), a,c,b), b,a,c), b,c,a), c,a,b), c,b,a)

  final def compare(der: Kiste) =
    if(a != der.a) a - der.a
    else if(b != der.b) b - der.b
    else if(c != der.c) c - der.c
    else vergleichInhalt(der)
  protected def vergleichInhalt(der: Kiste): Int

  final override def toString = baumString(0)
  private def baumString(lvl: Int): String = {
    val pref = " " * lvl + (if(lvl > 0) "\\-" else "")
    val selbst = getClass.getSimpleName + (a,b,c)
    pref + selbst + (if(kinder.isEmpty) "" else "\n") +
      kinder.map(_.baumString(lvl + 1)).
             mkString("\n")
  }

  override def equals(that: Any) = that match {
    case der: Kiste =>
      getClass() == der.getClass() &&
        a == der.a && b == der.b && c == der.c
    case _ => false
  }
  override def hashCode = 31*(31*(31*(31*(31 + v) + a) + b) + c)

  def finde(f: Kiste => Boolean): List[Kiste] =
    if(f(this)) this :: Nil
    else {
      val kinderPfade = kinder.map(_.finde(f))
      kinderPfade.find(!_.isEmpty) map {
        path => this :: path
      } getOrElse Nil
    }

  def kinder: Seq[Kiste]
  def istLeer = kinder.isEmpty
  def alsLeer = KisteLeer(a,b,c)
  def toStream: Stream[Kiste] =
    Stream(this).append(kinder flatMap (_.toStream))

  def *(anzahl: Int) = List.fill(anzahl)(this)
}

case class KisteLeer private[kisten](a: Int, b: Int, c: Int)
  extends Kiste {

  def kinder = Seq.empty

  def +(der: Kiste) = KisteHalb(a,b,c, der)
  protected def +<<(der: Kiste) = Set(KisteHalb(a,b,c, der))

  protected def vergleichInhalt(der: Kiste) = der match {
    case _: KisteLeer => 0//Zwei leere Kisten gleicher Größe sind die gleichen
    case _ => -1               //Sonst muss dieser Leere als Kleinerer weichen
  }

  override val hashCode = // Nutze den Companion, für die Hashfunktion.
    31*(31*(31* + super.hashCode) + KisteLeer.hashCode)

  // Überschreibe wegen Types!
  override def alsLeer = this
  override def *(anzahl: Int) = List.fill(anzahl)(this)
}

case class KisteHalb
  private[kisten](a: Int, b: Int, c: Int, links: Kiste) extends Kiste {

  override def istLeer = false
  def kinder = Seq(links)
  def freiFür(der: Kiste) = (false tuple_/: links) {
      (prev, x,y,z) => prev ||
        der < (a-x,b,c) ||
        der < (a,b-y,c) ||
        der < (a,b,c-z)
      }

  def ersetzeLinks(nl: Kiste) = KisteHalb(a,b,c, nl)

  def +(der: Kiste): KisteVoll = Kiste(a,b,c, links, der)

  protected def +<<(der: Kiste) = {
      val neueLinks = links +< der
      val kisten: Set[Kiste] = neueLinks.map(ersetzeLinks(_))
      if(this freiFür der)
        kisten + (this + der)
      else
        kisten
  }
  protected def vergleichInhalt(der: Kiste) = der match {
      case leer: KisteLeer => 1       // Ein leerer! Ach wie Tolle!
      case halb: KisteHalb =>
        this.links compare halb.links // !Hier! spielt der Inhalt eine Rolle!
      case _ => -1                    // Schau! Da bleibt nur noch der Volle..
    }

  override val hashCode = // wand sollte nie <0 sein
    31*(31*(31*(31* + super.hashCode) + links.hashCode) + KisteHalb.hashCode)
}

case class KisteVoll
  private[kisten](a: Int, b: Int, c: Int, links: Kiste, rechts: Kiste) extends Kiste {

  override def istLeer = false
  def kinder = Seq(links, rechts)

  def ersetzeLinks(nl: Kiste)  = Kiste(a,b,c, nl, rechts)
  def ersetzeRechts(nr: Kiste) = Kiste(a,b,c, links,  nr)

  private val linksGleichRechts = links == rechts
  protected def +<<(der: Kiste) = {
    val linkeSeite: Set[Kiste] = (links +< der).map(ersetzeLinks(_))
    if(linksGleichRechts)
      linkeSeite
    else {
      val rechteSeite = (rechts +< der).map(ersetzeRechts(_))
      linkeSeite ++ rechteSeite
    }
  }
  protected def vergleichInhalt(der: Kiste) = der match {
    case voll: KisteVoll =>                             // Der Inhalt entscheide!
      val linksDiff = this.links compare voll.links // Prüf erst die linke Seite!
      if(linksDiff != 0) linksDiff             // Sind es auch nicht die gleichen?
      else this.rechts compare voll.rechts     // Dann müssen die Rechten reichen
    case _ => 1                           // Sonst muss der kleine and're weichen!
  }

  override val hashCode =
    31*(31*(31*(31*(31 + super.hashCode) + links.hashCode) + rechts.hashCode)
        + KisteVoll.hashCode)
}

object Kiste {
  def ordne(x: Int, y: Int, z: Int) = {
    var (a,b,c) = (x,y,z)
    var tmp = 0
    if(a < b) { tmp = a; a = b; b = tmp }
    if(a < c) { tmp = a; a = c; c = tmp }
    // es gilt jetzt: a >= b && a >= c
    if(b < c) { tmp = b; b = c; c = tmp }
    // es gilt jetzt: a >= b >= c
    // also wenn eine Zahl nicht positiv ist, dann auf jeden Fall auch c
    if(c < 0)
     throw new IllegalArgumentException("Negative Werte sind nicht erlaubt!")
    else
      (a,b,c)
  }
  // Objekterzeuger [Hilfs-]Methoden:
  def apply(x: Int, y: Int, z: Int) = {
    val (a,b,c) = ordne(x,y,z)
    new KisteLeer(a,b,c)
  }
  def apply(x: Int, y: Int, z: Int, links: Kiste) = {
    val (a,b,c) = ordne(x,y,z)
    new KisteHalb(a,b,c, links)
  }
  
  def apply(x: Int, y: Int, z: Int, links: Kiste, rechts: Kiste) = {
    val (a,b,c) = ordne(x,y,z)
    // links muss >= rechts sein!
    if(links >= rechts) new KisteVoll(a,b,c, links, rechts)
    else                new KisteVoll(a,b,c, rechts, links)
  }

  object Ordnung {
    object nachVolumen extends Ordering[Kiste] {
      def compare(dieser: Kiste, anderer: Kiste) = dieser.v - anderer.v
    }
    object eindeutig extends Ordering[Kiste] {
      def compare(dieser: Kiste, anderer: Kiste) = dieser compare anderer
    }
  }

  implicit def intToPartialKiste(a: Int): PartialKisteA = PartialKisteA(a)
}

// Helper für schöne Syntax:
case class PartialKisteA(a: Int) {
  def x(b: Int) = PartialKisteAB(a,b)
}
case class PartialKisteAB(a: Int, b: Int) {
  def x(c: Int) = Kiste(a,b,c)
}
\end{lstlisting}

\subsection{Kistensatz}
\begin{lstlisting}
package de.voodle.tim.bwinf.kisten

import scala.collection.immutable.SortedSet
import scala.collection.immutable.TreeMap

case class Kistensatz (kistenBaum: TreeMap[Kiste, Int], v: Int, length: Int)
  extends Ordered[Kistensatz] {
  import Kistensatz._

  def find(f: Kiste => Boolean): List[Kiste] =
    kistenBaum.find(k => f(k._1)) map (_._1 :: Nil) getOrElse // Gibt es Wurzel?
    (((None: Option[List[Kiste]]) /: kistenSet) { // Sonst suche in Elementen.
      (vorher, kiste) =>
        if(vorher.isEmpty) {
          val pfad = kiste.finde(f)
          if(pfad.isEmpty)
            None
          else
            Some(pfad)
        } else vorher
    } getOrElse Nil)

  def -(der: Kiste) = Kistensatz(subFromTree(kistenBaum, der), v-der.v, length-1)
  def +(der: Kiste) = Kistensatz(  addToTree(kistenBaum, der), v+der.v, length+1)
  
  def +<(der: Kiste): Set[Kistensatz] =
    (Set[Kistensatz]() /: kistenBaum) {
        case (saetze, (k, _)) => {
          val treeOhneK = subFromTree(kistenBaum, k)
          val gepackt = (k +< der)
          val ms = gepackt.map(g =>
            Kistensatz(addToTree(treeOhneK, g), v, length)) // k.v == g.v
          saetze ++ ms
        }
      }
  def ++<(der: Kiste): Set[Kistensatz] =
    (this +< der) + (this + der)

  def neben(der: Kistensatz) =
    new AneinandergereihterKistensatz(der :: this :: Nil)

  def compare(der: Kistensatz) = Kistensatz.Ordnung.eindeutig.compare(this, der)

  override def equals(other: Any) =
    other.isInstanceOf[Kistensatz] && equals(other.asInstanceOf[Kistensatz])
  def equals(der: Kistensatz) = // Schneller HashCode basierter Check
    hashCode == der.hashCode && kistenBaum == der.kistenBaum

  override val hashCode = 41* (43 /: kistenBaum) { 47* _ + _.hashCode } + v

  def kistenSet: SortedSet[Kiste] = SortedSet(kistenBaum.map(_._1).toSeq: _*)
  def kisten = kistenBaum.flatMap((ki) => ki._1 * ki._2).toSeq.sorted
  def toStream = kisten.flatMap(_.toStream).toStream
  override def toString = kisten.mkString("{\n", ",\n", "\n}")
}

object Kistensatz {
  def addToTree(baum: TreeMap[Kiste, Int], karton: Kiste) =
    baum.updated(karton, baum.getOrElse(karton, 0) + 1)
  
  def subFromTree(baum: TreeMap[Kiste, Int], karton: Kiste) =
    baum.get(karton) match {
      case Some(i) => if(i > 1) baum.updated(karton, i-1) else (baum - karton)
      case _ => baum
    }
  private def createTree(tree: TreeMap[Kiste, Int], liste: Seq[Kiste]): TreeMap[Kiste, Int] =
    if(liste.isEmpty) tree
    else createTree(addToTree(tree, liste.head), liste.tail)

  def apply(kisten: Seq[Kiste]): Kistensatz =
      apply(createTree(TreeMap.empty, kisten))
  def apply(kistenBaum: TreeMap[Kiste, Int]): Kistensatz =
    Kistensatz(kistenBaum,
              (0 /: kistenBaum) { (v,ki) => v + ki._1.v * ki._2 },
              (0 /: kistenBaum) { (n,ki) => n + ki._2 } )

  object Ordnung {
    val nachVolumen = new Ordering[Kistensatz]() {
      def compare(x: Kistensatz, y: Kistensatz) = x.v compare y.v
    }
    val eindeutig = new Ordering[Kistensatz]() {
      def compare(x: Kistensatz, y: Kistensatz) =
        if(x.v != y.v) x.v - y.v
        else compare(x.kisten, y.kisten)

      private def compare(x: Iterable[Kiste], y: Iterable[Kiste]): Int =
        if(x.isEmpty && y.isEmpty) 0
        else if(x.isEmpty) -1
        else if(y.isEmpty) 1
        else {
          val diff = x.head compare y.head
          if(diff == 0) compare(x.tail, y.tail)
          else diff
        }
    }
  }
}
\end{lstlisting}
\subsection{Kistenpacker}
\begin{lstlisting}
package de.voodle.tim.bwinf.kisten

trait Kistenpacker {
  def min(kisten: Seq[KisteLeer]): Kistensatz
  // Sortiert wird von Groß nach Klein!
  // D'rum muss die Ordnung 'falsch'rum sein
  def sortiere(input: Seq[KisteLeer]): List[KisteLeer] =
    input.toList.sorted(Kiste.Ordnung.nachVolumen.reverse)
}
\end{lstlisting}
\subsubsection{OptimalPacker}
\begin{lstlisting}
package de.voodle.tim.bwinf.kisten

case object OptimalPacker extends Kistenpacker {
  def min(kistenListe: Seq[KisteLeer]) =
    packe(kistenListe) min Kistensatz.Ordnung.nachVolumen
 
  def packe(kistenListe: Seq[KisteLeer]) =
    (Set[Kistensatz]() /: sortiere(kistenListe)) ( packSchritt )

  protected def packSchritt(sätze: Set[Kistensatz], kiste: KisteLeer) =
    if(sätze.isEmpty)
      Set(Kistensatz(kiste :: Nil)) // KistenSatz nur mit der Kiste
    else for { satz <- sätze
               neuerSatz <- satz ++< kiste } yield neuerSatz
}
\end{lstlisting}
\subsubsection{AufteilenderPacker}
\begin{lstlisting}
package de.voodle.tim.bwinf.kisten

case class AufteilenderPacker (t: Int = 15) extends Kistenpacker {
  def min(kisten: Seq[KisteLeer]): Kistensatz = {
    val kistengruppen = teileKisten(kisten)
    val kistensätze = for {
      kistengruppe <- kistengruppen
    } yield OptimalPacker min kistengruppe
    
    kistensätze reduceLeft { _ neben _ }
  }

  private def teileKisten(kisten: Seq[KisteLeer]) = {
    val m = (kisten.length.toDouble / t).ceil.toInt
    val startListen = Vector.fill(m)(List.empty[KisteLeer])
    ((startListen, 0) /: kisten) {
      case ((listen, idx), kiste) =>
        val liste = listen(idx)
        (listen updated (idx, kiste :: liste), (idx+1)%m)
    } match {
      case (listen, _) => listen
    }
  }
}
\end{lstlisting}
\subsubsection{OfflinePacker und OnlineAlgo}
\begin{lstlisting}
package de.voodle.tim.bwinf.kisten

/** Gliedert den OnlineAlgo in die Packer Hierarchie */
case class OfflinePacker(onlinePacker: OnlineAlgo, sortieren: Boolean)
  extends Kistenpacker {
  def min(kistenSeq: Seq[KisteLeer]) = {
    val kisten = if(sortieren) sortiere(kistenSeq) else kistenSeq
    (onlinePacker ++ kisten).kisten
  }
}
object OfflinePacker {
  def apply(strategien: List[Strategie], sortieren: Boolean = true): OfflinePacker =
    OfflinePacker(OnlineAlgo(strategien), sortieren)
  def apply(sortieren: Boolean): OfflinePacker =
    OfflinePacker(OnlineAlgo.standardStrategien, sortieren)
  def apply(): OfflinePacker =
    OfflinePacker(OnlineAlgo.standardStrategien)
}
case class OnlineAlgo(kisten: Kistensatz, strategien: List[Strategie]) {
  /** Einfach alle nacheinander hineinfügen! */
  def ++ (die: Seq[KisteLeer]) = (this /: die)(_ + _)

  def + (der: KisteLeer): OnlineAlgo = {
    val neueKisten = (Option.empty[Kistensatz] /: strategien) {
        // wenn es  vorher gibt, dann vorher, sonst diese strategie benutzen!
         (vorher, strat) => vorher orElse strat(kisten)(der)
      } getOrElse (kisten + der) // sonst aneinanderreihen, wie einfach :D
    OnlineAlgo(neueKisten, strategien)
  }
}
object OnlineAlgo { // Hilfsmethoden zum schnellen erzeugen! :)
  def apply(): OnlineAlgo = apply(Kistensatz(Nil))
  def apply(strategien: List[Strategie]): OnlineAlgo =
    OnlineAlgo(Kistensatz(Nil), strategien)
  def apply(kistenSatz: Kistensatz): OnlineAlgo =
    OnlineAlgo(kistenSatz, standardStrategien)

  def standardStrategien =
    FindeGrößerenLeeren :: // (2)
    FindeZwischenraum   :: // (3)
    FindeHalbleeren     :: // (1)
    FindeKleinereWurzel :: // (4)
    Nil
}
\end{lstlisting}

\subsubsection*{Strategien}

\begin{lstlisting}
package de.voodle.tim.bwinf.kisten

abstract trait Strategie {
  // Syntactic sugar :)
  def apply(kisten: Kistensatz)(der: KisteLeer) =
    finde(kisten)(der)

  protected def finde(kisten: Kistensatz)(der: KisteLeer): Option[Kistensatz]
}

object Strategie {
  private[kisten] def pfadErsetzer(pfad: List[Kiste]): (Kiste,Kiste) => Kiste = {
      case (kg: KisteHalb, kl) => kg ersetzeLinks kl
      case (kg: KisteVoll, kl) =>
        if(pfad.contains(kg.links)) kg ersetzeLinks  kl
        else                        kg ersetzeRechts kl
      case _ =>
        throw new IllegalArgumentException("Leere Kisten im Pfad nicht erlaubt!")
    }
}
import Strategie._

case object FindeHalbleeren extends Strategie {
  protected def finde(kisten: Kistensatz)(der: KisteLeer) =
    kisten.find {
      case kh: KisteHalb =>
        (kh >~ der) && kh.freiFür(der)
      case _ => false
    } match {
      case Nil => None
      case pfad =>
        val alteKiste = pfad.head
        val neueKiste = (pfad :\ (der: Kiste)) { (_, _) match {
            case (kg: KisteHalb, kl: KisteLeer) => kg + kl
            case (k1, k2) => pfadErsetzer(pfad)(k1, k2)
          }}
        Some(kisten - alteKiste + neueKiste)
    }
}
case object FindeGrößerenLeeren extends Strategie {
  protected def finde(kisten: Kistensatz)(der: KisteLeer) =
    kisten.find {
      big => (big >~ der) && big.istLeer
    } match {
      case Nil => None
      case pfad =>
        val alteKiste = pfad.head
        val neueKiste = (pfad :\ (der: Kiste)) { (_, _) match {
            case (kg: KisteLeer, kl: KisteLeer) => kg + kl
            case (k1, k2) => pfadErsetzer(pfad)(k1, k2)
          }}
        Some(kisten - alteKiste + neueKiste)
    }
}
case object FindeZwischenraum extends Strategie {
  protected def finde(kisten: Kistensatz)(der: KisteLeer) =
    kisten.find { k => (k >~ der) && (k match {
      case kh: KisteHalb =>//Es ist sicher noch Platz für einen Zwischenraum
	  (der >~ kh.links)//Aber kh.links muss auch in der passen!
      case kv: KisteVoll =>//Nur dann ist Platz, wenn 'der' noch neben den anderen reinpasst.
	val links  = Kiste(kv.a,kv.b,kv.c, kv.links)
	val rechts = Kiste(kv.a,kv.b,kv.c, kv.rechts)
	((der >~ kv.links)  && rechts.freiFür(der)) ||
	((der >~ kv.rechts) && links.freiFür(der))
      case _ => false
     })
    } match {
      case Nil => None
      case pfad =>
        val alteKiste = pfad.head
        val neueKiste = (pfad :\ (der: Kiste)) {
          case (kg: KisteHalb, kl: KisteLeer) =>
            kg ersetzeLinks (kl + kg.links)
          case (kg: KisteVoll, kl: KisteLeer) =>
            if(kl >~ kg.links) kg ersetzeLinks  (kl + kg.links)
            else               kg ersetzeRechts (kl + kg.rechts)
          case (k1, k2) => pfadErsetzer(pfad)(k1, k2)
        }
        Some(kisten - alteKiste + neueKiste)
    }
}
case object FindeKleinereWurzel extends Strategie {
  protected def finde(kisten: Kistensatz)(der: KisteLeer) =
    kisten.kistenSet.find(der >~ _) map {
      kWurzel => kisten - kWurzel + (der + kWurzel)
    }
}
\end{lstlisting}
\subsection{Helfer}
\subsubsection{Utils}
\begin{lstlisting}
package de.voodle.tim.bwinf.kisten

import java.math.RoundingMode.{UP => ROUND_UP}
import java.text.DecimalFormat

object Utils {
  import StandardKisten._ // Standardkartongrößen
  import KistenLogger._
  import LogStufe._
  import java.math.MathContext

  private val logger = KistenLogger(print, getNextFileWriter)
  import logger.logge

  private implicit val mathContext = new MathContext(4, ROUND_UP)

  def bench(packer: Kistenpacker, kistenFunktion: ()=>Seq[KisteLeer],
            wiederholungen: Int, verbose: Boolean = true): (BigDecimal, Long) = {
    logge("Packer: " + packer)
    var ergebnisse: List[((BigInt, BigInt), Long)] = Nil
    var i = 0
    while(i < wiederholungen) {
      val kisten = kistenFunktion()

      val (min, zeit) = bench(packer min kisten)
      logge("Minimum: " + min)(verbose)

      val v0 = kisten.map(i => BigInt(i.v)).sum
      val v1 = min.kisten.map(i => BigInt(i.v)) reduceLeft (_ + _)
      ergebnisse ::= ((v0,v1), zeit)
      i += 1
    }
    val gesamtVolumen = ergebnisse.map(_._1._1).sum
    val endVolumen    = ergebnisse.map(_._1._2) reduceLeft (_ + _)
    val komprimierung = BigDecimal(
      BigDecimal(gesamtVolumen).bigDecimal
        .divide(BigDecimal(endVolumen).bigDecimal, mathContext)
      )
    val zeiten = ergebnisse map (_._2)
    val zeit = zeiten.sum / zeiten.length
    val format = new DecimalFormat("##.##")
    logge("Wiederholungen\tKomprimierungsgrad\tZeit")
    logge(wiederholungen + "\t\t" + format.format(komprimierung) +
                           "\t\t\t" + format.format(zeit))
    (komprimierung, zeit)
  }

  private def zufallsKisten(n: Int) = List.fill(n)(zufall)

  // Hilfsmethode zum benchen.
  private def bench[T](packer: =>T): (T, Long) = {
    val startZeit = System.currentTimeMillis
    val kistenSaetze = packer
    val endZeit = System.currentTimeMillis
    (kistenSaetze, endZeit - startZeit)
  }
}
\end{lstlisting}
\subsubsection{StandardKisten}
\begin{lstlisting}
package de.voodle.tim.bwinf.kisten

import Kiste._ // Implicits importieren für schöne Syntax

object StandardKisten { // Alle Angaben in cm
  // Quelle: http://umzugskarton.de
  val umzugskarton  = 63 x 31 x  33
  val buecherkarton = 41 x 32 x  34
  val kleiderbox    = 50 x 60 x 135

  // Quelle: http://www.kartonfritze.de
  val pizzakarton = 24 x 24 x 4// .. auch wenn Frau Y. nicht Informatikfreak[in] ist..
  val arthur    = 46 x 38 x 29
  val dinodrei  = 43 x 35 x 10
  val fluschi   = 43 x 34 x 23
  val christoph = 42 x 33 x 37
  
  // "Ideale Kartongröße für Vertreter des weiblichen Geschlechts." laut Kartonfritze
  val paroli      =  61 x 34 x 35 
  val postmeister = 120 x 60 x 60 // Was machen jetzt Postmeisterinnen?
  val chachacha   = 118 x 20 x  9 // Waffenkarton: "Geeignet für alles was passt!"-ach nee!

  private def zufallGen = scala.util.Random
  def zufall: KisteLeer = zufall(60,60,60)
  def zufall(maxA: Int, maxB: Int, maxC: Int): KisteLeer =  // Größer 0!
    zufallGen.nextInt(maxA-1) + 1 x
    zufallGen.nextInt(maxB-1) + 1 x
    zufallGen.nextInt(maxC-1) + 1
  def zufallKisten(maxA: Int, maxB: Int, maxC: Int): Stream[KisteLeer] =
    Stream.cons(zufall(maxA,maxB,maxC), zufallKisten(maxA, maxB, maxC))
}
\end{lstlisting}
\subsubsection{KistenLogger}
\begin{lstlisting}
package de.voodle.tim.bwinf.kisten

object LogStufe extends Enumeration {
  type LogStufe = Value
  val Extra    = Value(20)
  val Ergebnis = Value(10)
}
object KistenLogger {
  import java.io.{File, FileWriter}
  val suffix = "_kisten.log"
  def getNextFileWriter =
    Stream from 0 map (i => new File("./" + i + suffix)) find {
      ! _.exists
    } map {
      file => new FileWriter(file)
    } getOrElse (throw new AssertionError("No file found!"))
  implicit def ausgabeAusFileWriter(fw: FileWriter): String => Any = {
    str => fw.write(str); fw.flush
  }
}
case class KistenLogger(printer: String => Any, dateiAusgabe: String => Any) {
  import LogStufe._
  def logge(zeile: =>String)(implicit print: Boolean = true) =
    if(print) {
      val ausgabenZeile = zeile + "\n" // By-name!
      printer(ausgabenZeile)
      dateiAusgabe(ausgabenZeile)
    } else ()
}
\end{lstlisting}
\clearpage

\setlength{\hoffset}{0cm}
\setlength{\voffset}{0cm}
\lstset{xleftmargin=0cm,xrightmargin=-0cm}
\lstset{basicstyle=\ttfamily}
\fontsize{10pt}{10.8pt}