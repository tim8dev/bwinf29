\section{Zweite bearbeitete Aufgabe: (2) Containerklamüsel}
\subsection{Lösungsidee}
\subsubsection{Vorüberlegungen}
Für die Lösung der Aufgabe werden bestimmte Eigenschaften von Permutationen zu Nutze gemacht. Zunächst lässt sich feststellen, dass
die Anordnung der Waggons zu den Containern eine bijektive Abbildung von $[1,n]$ nach $[1,n]$, sprich, eine Permutation der Menge $[1,n]$ ist.
Die entscheidende Eigenschaft die der von mir entwickelte Algorithmus nutzt ist die Tatsache, dass
sich jede Permutation als Folge von disjunkten Zyklen darstellen lässt.\\
Was ein Zyklus im Ungefähren ist und was er für die Aufgabe bedeutet, lässt folgende Darstellung vermuten.
Man beachte, dass die Container ``in einem Stück'' getauscht und an die richtige Position gebracht werden können.
\lstset{basicstyle=\ttfamily\small}
\begin{lstlisting}
1 2 3 4 5 6 7 8 ... (Index)
2 5 3 8 4 6 7 1 ... (Containernummer)
-->
  ------>
      <--
      -------->
<--------------
\end{lstlisting}
\lstset{basicstyle=\ttfamily\normalsize}
Um den Begriff eines Zyklus' genauer einzuführen, zitiere ich folgend Beutelspacher. \\

``Eine Permutation $\pi$ von $X$ wird ein \emph{Zyklus} [...] genannt, falls - grob gesprochen - die Elemente, die von $\pi$ bewegt werden,
zyklisch vertauscht werden. Genauer gesagt: Eine Permutation $\pi$ heißt zyklisch,
falls es ein $ i \in X$ und eine natürliche Zahl $k$ gibt, so dass die folgenden drei Bedingungen gelten:
\begin{enumerate}
 \item $\pi^k(i) = i$,
 \item die Elemente $i,\pi(i),\pi^2(i),\dots,\pi^{k-1}(i)$ sind paarweise verschieden,
 \item jedes Element, das verschieden von $i,\pi(i),\pi^2(i),\dots,\pi^{k-1}(i), \pi^k(i) (=i)$ ist, wird von $\pi$ fest gelassen.
\end{enumerate}
Die kleinste natürliche Zahl k mit obiger Eigenschaft wird die \emph{Länge} des Zyklus $\pi$ genannt.
Ein Zyklus der Länge k heißt auch k-Zyklus. Wir schreiben dann \[\pi=(i\;\pi(i)\;\pi^2(i) \; \dots \;\pi^{k-1}(i)).\]
[\dots]
\paragraph{Darstellung einer Permutation als Produkt disjunkter Zyklen.}
\emph{Jede Permutation kann als Produkt zyklischer Permutationen geschrieben werden, von denen keine zwei ein Element gemeinsam haben.}\\
\emph{Das heißt: Zu jedem $\pi \in S_n$ [$S_n$ ist die Menge aller Permutation von $[1,n]$ in sich.] gibt es zyklische Permutationen $\zeta_1,\dots,\zeta_s \in S_n$,
so dass folgende Eigenschaften erfüllt sind: \\
 - $\pi=\zeta_1\cdot \zeta_2 \cdot \ldots \cdot \zeta_s$ \\
 - kein Element aus $X$[$X=[1,n]$], das als Komponente in $\zeta_i$ vorkommt, kommt in $\zeta_j$ vor ($i,j=1, \dots,n, i \neq j$).
(Das bedeutet: Wenn ein Element $x \in X$ in einem Zyklus $\zeta_i$ ``vorkommt'',
 so wird $x$ von jedem anderen Zyklus $\zeta_j \: (j \neq i)$ fest gelassen.)
}''\footnote{Definitionen, Sätze und Erklärung übernommen aus Lineare Algebra, Albrecht Beutelspacher, S.174f}\\

Die Darstellung der Permutation als Produkt disjunkter Zyklen erweist sich als günstig,
denn nun kann das Problem in folgende zwei Teile aufgebrochen werden.
Der erste ist, die Container eines Zyklus' an die richtige Stelle zu bringen.
Dies lässt sich relativ leicht realisieren, indem der Container am Anfang eines Zyklus' an die richtige Position gebracht wird,
anschließend der zweite an die richtige Position, \dots, bis der Ausgangspunkt wieder erreicht ist.
Der zweite Teil besteht also darin, die Zyklenabarbeitung dort zu unterbrechen, wo eine andere beginnt.
Da nach der Abarbeitung des nächsten Zyklus der Kran wieder an der Position ist,
wo der erste Zyklus unterbrochen wurde, kann die Abarbeitung ``einfach'' fortgesetzt werden. \\
Etwas anders ausgedrückt:
Beginnend am Anfang eines Zyklus', können dessen Container ``in einem Stück'' an die richtige Stelle gebracht werden
 und der Kran kann anschließend wieder an der Ausgangsposition angelangen.
Wir werden etwas später sehen, dass dadurch tatsächlich auch immer ein optimaler Weg (zumindest innerhalb eines Zyklus) gefunden werden kann.
Durch entsprechend richtige ``Konkatenation'' bzw. ``Verschachtelung'' der Befehlsketten für die einzelnen Zyklen lässt sich immer
 ein nach dem in der Aufgabenstellung vorgegebenem Gütekriterium optimaler Weg des Krans erstellen.
Der durch Ausführung der berechneten Instruktionen abzufahrende Weg ist also minimal. \\

Vor der folgenden, eigentlichen Dokumentation möchte ich noch von mir verwendete Konventionen erläutern.
Betrachte ich in einem Teil der Dokumentation Code, bzw. Codeausschnitte,
benutze ich |monospaced Font| und die im Code selber benutzten Bezeichner um diese darzustellen. \\
Zur Erläuterung mathematischer Überlegungen wird der $math-Mode$ von \TeX benutzt.
Statt beispielsweise |perm| oder |cycle| als Bezeichner im ``code-Mode'' benutze ich $\pi$ respektive $\zeta$ im ``math-Mode''.
Es werden also die jeweils passenderen Bezeichner und Symbole verwendet.
\subsubsection{Datenstruktur}
Permutationen auf $[1,n]$ können in einer indexierten Liste jeder Art (beispielsweise einem Array) gespeichert werden.
Da in der Informatik jedoch indexierte Listen (insbesondere Arrays) meist Indizes aus $[0,n[$ besitzen muss dies beim Zugriff beachtet werden.
Der Definitionsbereich ist also um eins ``nach links'' verschoben.
Um also die Zahl |p| zu finden, auf die |i| durch |perm| abgebildet wird, gilt |p = perm(i-1)| und nicht |p = perm(i)|. \\
Es wird außerdem noch eine einfache Datenstruktur benötigt, um das Gleis mit Containerstellplätzen und Waggons abzubilden.
Hierfür werden zwei Arrays verwaltet, die zu jedem Index den Container speichern der gerade auf dem Containerstellplatz bzw. Waggon steht.
Die Implementierung dieser Datenstruktur wird in \ref{gleis} noch genauer erläutert.
%\newpage
\subsubsection{Ergebnisoptimaler Algorithmus}
In diesem Abschnitt wird zunächst ein Algorithmus entworfen, der optimale Ergebnisse (im Sinne von kürzesten Kranwegen) berechnet.
Anschließend wird das Laufzeitverhalten dieses Algorithmus betrachtet, welches gut, jedoch nicht bestmöglich ist.
Im darauffolgendem Abschnitt wird ein Algorithmus vorgestellt, der sowohl optimale Ergebnisse berechnet als auch optimale Laufzeitkomplexität vorweist.
\paragraph{Entwurf}
\label{slow_cycler}
Der Entwurf dieses Algorithmus' ergibt sich aus den obigen Überlegungen. Zunächst wird die Zerlegung in disjunkte Zyklen berechnet.
Hierfür wird folgende Hilfsfunktion zur Berechnung \emph{eines} Zyklus' verwendet.
Wichtig ist hierbei zu beachten, dass die Waggonnummer an der Stelle |idx| durch |perm(idx-1)| dargestellt wird.\\
Salopp gesagt, hangelt man sich so lange - bei einem Startindex beginnend - durch die Permutation, bis man wieder beim Anfangswert ankommt.
Genauer betrachtet, liefert die Unterfunktion |step| die verbleibenden Zahlen des Zyklus' nach |idx|.
|step| bricht mit der leeren Liste ab, wenn |start| wieder erreicht wird.
Andernfalls reicht |step| den aktuellen Wert |idx| vor die restlichen - rekursiv durch |step| - berechneten Zahlen.
|cycle| braucht nur mehr |step| mit |start| aufzurufen.
Das Ergebnis von |cycle| ist also ein Zyklus der oben beschrieben Form $i, \pi(i), \pi^2(i), \dots, \pi^{k-1}(i)$
als |start :: perm(start-1) :: perm(perm(start-1)-1) :: |\dots| :: Nil|
\lstset{language=Scala}
\lstset{basicstyle=\ttfamily\normalsize}
\begin{lstlisting}
def cycle(perm: Seq[Int], start: Int): List[Int] = {
  def step(idx: Int): List[Int] =
    if(start == idx) Nil
    else idx :: step(perm(idx - 1))
  step(start)
}
\end{lstlisting}
Nun lässt sich auch recht einfach ein Algorithmus zum Finden der disjunkten Zyklen einer Permutation angeben.
Die folgend dargestellte rekursive Funktion |cyclesOf| liefert eine Liste von disjunkten Zyklen (also eine Liste von Listen von Zahlen)
die die Permutation darstellen. Um disjunkte Zyklen zu finden, müssen sich jeweils alle bisher abgearbeiteten Zahlen gemerkt werden.
Dies erfolgt in einem |Set| (standardmäßig ein |HashSet| in Scala). \\
In jedem Rekursionsschritt wird zunächst der neue Startwert |start| gesucht.
Der Startwert ist die erste Zahl von |1| bis |perm.length| die noch nicht abgearbeitet wurde (also nicht in |ready| enthalten ist).
Anschließend wird der neue Zyklus |newCycle| mit der Hilfsfunktion |cycle| berechnet.
Dann wird die neue Menge aller abgearbeiteten Zahlen |newReady| gebildet, indem alle Zahlen aus |newCycle| in |ready| eingefügt werden.
Zuletzt erfolgt der rekursive Aufruf, wobei |newCycle| vor den rekursiv berechneten Zyklen angefügt wird.
% TODO: Wording.
Die Rekursion wird abgebrochen, sobald alle Zahlen abgearbeitet wurden. Dies lässt sich daran erkennen,
dass keine Zahl mehr in |1|\dots|perm.length| gefunden werden kann, die noch nicht abgearbeitet - also in |ready| enthalten - ist.
Es wird dann die leere Liste |Nil| zurückgegeben.
\lstset{language=Scala}
\lstset{basicstyle=\ttfamily\normalsize}
\begin{lstlisting}
def cyclesOf(perm: Seq[Int], ready: Set[Int]): List[List[Int]] =
  (1 to perm.length) find (i => !ready.contains(i)) match {
      case Some(start) =>
        val newCycle = cycle(perm, start)
        val newReady = ready ++ newCycle // O(n)
        newCycle :: cyclesOf(perm, newReady)
      case None =>
        Nil
    }
\end{lstlisting}
Anhand der berechneten Zyklen wird im nächsten Schritt die Instruktionskette errechnet.
Hierfür wird zunächst die folgend abgebildete Methode |computeFromCycles| definiert,
welche sich einer - gleich anschließend betrachteten, - weiteren Funktion |computeCycle| bedient.
\lstset{basicstyle=\ttfamily\small}
\begin{lstlisting}
def computeFromCycles(cycles: Cycles): Seq[Instruction] = {
  // Füge erstes TakeCon hinzu, damit bereits ein Container auf dem Kran ist;
  // Lösche letzes Element (PutWag) mit init
  TakeCon :: computeCycle(cycles.head, cycles.tail).init._1.toList
}
\end{lstlisting}
Im Allgemeinen soll die Funktion |computeCycle| für einen Zyklus |cycle| und die restlichen Zyklen |other| die Instruktionen für einen Weg liefern,
so dass alle Elemente der gegebenen Zyklen an die richtige Position gebracht werden und der Kran wieder an die Startposition gebracht wird.
Hierbei geht |computeCycle| davon aus, dass bereits der 1. Container des Zyklus auf den Kran gehoben wurde und noch kein anderer Container des Zyklus' bewegt wurde.
Außerdem werden Container immer auf der Containerseite transportiert.
Das Grundprinzip des Algorithmus' ist es, dass nacheinander alle Elemente des Zyklus abgearbeitet werden.
Dazu wird ein |foldLeft| über den Zyklus ausgeführt.
Damit der Kran auch wieder zum Ersten Element (Startposition) zurückgefahren wird, wird dieses an den Zyklus angehängt.
In jedem Schritt werden die bisherigen Instruktionen, die verbleibenden Nachfolgerzyklen und das vorherige Element übergeben.
Es wurde eine Hilfsmethode |step| geschrieben, an die die Parameter übergeben werden. \\
Im folgenden ist der Code dargestellt nur mit Deklaration aber ohne Definition der Hilfsfunktion |step|.
(Ich entschied mich zwecks Lesbarkeit und Strukturierung, den Code in mehrere Teile aufzuteilen.)
Bemerkung: Um den Code gut in der Dokumentenzeilenbreite darstellen zu können,
wurden die sogenannten ``type aliases'' |Cycle| für |List[Int]|, |Cycles| für |List[Cycle]| und,
speziell für die Hilfsfunktion |step| |Step| für |(ListBuffer[Instruction], Cycles, Int)|.
\begin{lstlisting}
def computeCycle(cycle: Cycle,
                 other: Cycles): (ListBuffer[Instruction], Cycles) = {
  val max = cycle.max

  type Step = (ListBuffer[Instruction], Cycles, Int)
  def step(instrs: ListBuffer[Instruction], cyclesLeft: Cycles,
           prev: Int, cur: Int): Step = [...]

  val erster = cycle.head
  val initial = (ListBuffer[Instruction](), other, erster)
  // Arbeite alle Elemente des Zyklus' ab
  val (instrs, cyclesLeft, last) = (cycle.tail :+ erster).foldLeft(initial) {
    case ((instrs, cyclesLeft, prev), cur) =>
      step(instrs, cyclesLeft, prev, cur)
  }
  (instrs, cyclesLeft)
}
\end{lstlisting}
Die Hilfsfunktion |step| unterscheidet 3 Fälle (diese sind hinter den |case| Anweisungen in Klammern in den Kommentaren im nachfolgendem Codeausschnitt markiert).
\begin{enumerate}
 \item Wenn das letzte betrachtete Element |prev| das Maximum |max| ist und es ein nächsten Zyklus |nextCycle| gibt,
       dessen erstes Element |next| eins weiter rechts von |max| bzw. |prev| ist, dann konkateniere die Zyklen entsprechend.
       Das heißt, es wird erst mit |computeCycle| die Instruktionen |cycleInstrs| für die nächsten Zyklen berechnet
       und diese anschließend angehängt. Zudem müssen ein paar wenige Instruktionen ``zwischen'' den Zyklen,
       also vor und nach |cycleInstrs| generiert werden. Anschließend wird |step| nochmals aufgerufen,
       diesmal mit den neuen Instruktionen |extraInstrs| und keinen restlichen Zyklen, da diese bereits alle abgearbeitet sind.
 \item Wenn das aktuell betrachtete Element |cur| größer ist, als das erste Element |next| des nächsten Zyklus |nextCycle|,
       dann wird zunächst der nächste Zyklus abgearbeitet.
       Hierfür wird |computeCycle| mit |nextCycle| und den restlichen Zyklen |cyclesLeft.tail| aufgerufen.
       Hierbei können Zyklen ``übrig'' bleiben, nämlich wenn das maximale Element des Zyklus |next| kleiner ist als das maximale Element |max| dieses Zyklus |cycle|.
       Diese möglicherweise ``übrig'' geliebenen Zyklen |newCyclesLeft| werden zusammen mit den neuen Instruktionen wieder an |step| übergeben.
 \item Wenn weder der 1. noch der 2. Fall zutrifft, werden lediglich Instruktionen generiert,
       die den Kran von |prev| nach |cur| bewegt, aktuellen Container auf den Waggon ablegt und dann den Container auf dem Containerstellplatz aufhebt.
\end{enumerate}
Anschließend werden die durch das |foldLeft| erzeugten Instruktionen und restlichen Zyklen zurück\-ge\-ge\-ben.\\
Folgend ist der Scalacode abgebildet, welcher den Algorithmus zur Berechnung der Instruktionen darstellt.
Dieser Codeausschnitt ist deutlich komplexer als die vorherigen. Deswegen wurden Kommentare und Markierungen hinzugefügt.
\lstset{basicstyle=\ttfamily\small}
\begin{lstlisting}
type Step = (ListBuffer[Instruction], Cycles, Int)

def step(instrs: ListBuffer[Instruction], cyclesLeft: Cycles,
         prev: Int, cur: Int): Step =
  cyclesLeft.headOption match {
   // Gibt es ein nächsten Zyklus und beginnt dieser direkt nach diesem?
   case Some(nextCycle @ (next :: _)) if prev == max && max+1 == next =>//(1)
     // Wenn ja, "konkateniere" diese.
     val (cycleInstrs, _) = computeCycle(cyclesLeft.head, cyclesLeft.tail)
     val extraInstrs = instrs ++=
       ListBuffer(PutCon, MoveRight, TakeCon) ++=
       cycleInstrs ++= ListBuffer(MoveLeft, TakeCon)
     step(extraInstrs, Nil, prev, cur)
   // Gibt es ein nächsten Zyklus und beginnt dieser vor dem nächsten Element?
   case Some(nextCycle @ (next :: _)) if cur > next => //(2)
     // Wenn ja, dann arbeite erst nextCycle ab.
     val (cycleInstrs, newCyclesLeft)=computeCycle(nextCycle, cyclesLeft.tail)
     val newInstrs = instrs ++=
       ListBuffer(Move(prev -> next), Rotate, TakeCon,
                                      Rotate, PutCon, Rotate) ++= cycleInstrs
     step(newInstrs, newCyclesLeft, next, cur)
   case _ => //(3)
     // Andernfalls, fahre einfach mit der Abarbeitung fort.
     val newInstrs = instrs ++=
       ListBuffer(Move(prev -> cur), Rotate, PutWag, TakeCon)
     (newInstrs, cyclesLeft, cur)
  }
\end{lstlisting}
\lstset{basicstyle=\ttfamily\normalsize}

\paragraph{Optimale Ergebnisse}
\label{opt_res}
Dieser Algorithmus liefert bereits optimale Ergebnisse im Sinne des Güte\-kri\-ter\-iums der Aufgabenstellung.
Um dies zu zeigen, wird zunächst bewiesen, dass die Zyklen richtig gefunden werden. \\

Zunächst wird die Korrektheit der Hilfsfunktion |cycle| gezeigt.
Das heißt, wir vergewissern uns, dass |cycle| zu einer gegebenen Permutation |perm| immer einen Zyklus findet, der an dem Startindex |start| beginnt.
Da ich nachfolgend nun mathematisch Argumentieren möchte, ersetzte ich die Programmbezeichnungen durch mathematische.
Konkret stelle ich |perm| durch $\pi$, den gesuchten Zyklus durch $\zeta$ und |start| durch $i$ dar.
Es ist also ein Zyklus $\zeta$ der folgenden Form gesucht.
\[ \zeta = (i, \zeta(i),\zeta^2(i), \dots, \zeta^{k-i}(i)) \]
Für alle $x$ die im Zyklus $\zeta$ enthalten sind gilt $\zeta(x) = \pi(x)$.
Weiter sind genau die Elemente \linebreak $i, \zeta(i),\zeta^2(i), \dots, \zeta^{k-i}(i)$ enthalten, also gilt
\[ \zeta = (i, \zeta(i),\zeta^2(i), \dots, \zeta^{k-i}(i)) = (i, \pi(i),\pi^2(i), \dots, \pi^{k-1}(i)) \]
Nun betrachten wir nochmals die Funktionsweise von |cycle|, bzw. von |step|.
Wir behaupten zunächst |step| liefert zu einer Zahl $j = \pi^x(i)$ die Zahlen $\pi^x(i), \pi^{x+1}(i), \dots, \pi^{k-1}(i)$.
Dies machen wir uns durch Induktion über $x$ klar. Sei also $x = k$.
Dann gilt nach Definition eines Zyklus' $j = \pi^x(i) = \pi^k(i) = i$, also bricht |step| hier ab und liefert die leere Liste, was in der Tat korrekt ist.
Nun können wir annehmen, |step| liefert für ein $j = \pi^x(i)$ bereits die Zahlen $\pi^x(i), \pi^{x+1}(i), \dots, \pi^{k-1}(i)$.
Also zeigen wir nun, dass |step| auch für ein $l = \pi^{x-1}$ die richtigen Zahlen liefert.
|step| reiht also $l$ vor die Zahlen, die durch Aufruf von |step| mit $\pi(l) = \pi(\pi^{x-1}) = \pi^x = j$ berechnet werden.
Das ergibt genau die Zahlen $\pi^{x-1}(i), \pi^x(i), \pi^{x+1}(i), \dots, \pi^{k-1}(i)$.
Die Aussage ist somit bewiesen.
Wird nun |step| - wie in |cycle| - mit $j = \pi^0(i) = i$ aufgerufen, erhalten wir korrekterweise die Zahlen
\[ (\pi^x(i), \pi^{x+1}(i), \dots, \pi^{k-1}(i)) = (\pi^0(i), \pi^1(i), \dots, \pi^{k-1}(i)) = (i, \pi(i), \pi^2(i), \dots, \pi^{k-1}(i)) = \zeta.\]

Im Folgenden können wir uns also der Korrektheit von |cycle| sicher sein. Nun soll die Korrektheit von |cyclesOf| gezeigt werden.
Auch hier wähle ich mathematische Symbole/Bezeichner. Die Liste der disjunkten Zyklen, die |cyclesOf| berechnen soll bezeichne ich mit $\zeta_1, \zeta_2, \dots, \zeta_o$.
Wir wollen also beweisen, dass |cyclesOf| zu einer gegebenen Permutation $\pi$ und einer leeren Menge von ``fertigen'' Elementen eine Liste
von disjunkten Zyklen $\zeta_1, \zeta_2, \dots, \zeta_o$ zurückgibt, wobei $o$ die Anzahl disjunkter Zyklen ist und $x < y \Leftrightarrow \ min(\zeta_x) < min(\zeta_y)$ für alle $x,y=0 \dots o$.
Es sollen also nach Startwert sortierte Zyklen zurückgeliefert werden.
Es wird im folgenden wieder Induktion verwendet.
Im Induktionsanfang soll also gezeigt werden, dass |cyclesOf| für $ready = \zeta_1 \cup \zeta_2 \cup \dots \cup \zeta_o$ alle verbleibende Zyklen - nämlich gar keine - findet.
Da $\|ready\| = \|\zeta_1 \cup \dots \cup \zeta_o\|$, bricht |cyclesOf| ab mit der leeren Liste. Dies ist korrekt, denn es sind bereits alle Zyklen gefunden.
Nun gelte, dass |cyclesOf| für ein $x \in \{0, \dots, o\}$ und $ready = \zeta_1 \cup \zeta_2 \cup \dots \cup \zeta_{x-1} \cup \zeta_x$ die Zyklen $\zeta_{x+1}, \dots, \zeta_o$ findet.
Wir zeigen, dass dies auch für $x \rightarrow x-1$ gilt.
Zunächst wird der Wert $s \in \{1, \dots, n\}$ ($s=$|start|, $n$ ist die Länge von $\pi$) mit $s \notin ready$ gesucht.
Nun wird der neue Zyklus $\zeta_x$ berechnet. Dieser ist sicher disjunkt von den zuvor berechneten Zyklen, da er bei $s \notin ready$ beginnt.
Anschließend wird |cyclesOf| rekursiv aufgerufen, mit |ready| = $\zeta_1 \cup \dots \cup \zeta_{x-1} \cup \zeta_{x}$.
Dieser Aufruf liefert nach Induktionsannahme die Zyklen $\zeta_{x+1}, \dots, \zeta_o$.
Also werden insgesamt die Zyklen $\zeta_{x}, \zeta_{x+1}, \dots, \zeta_o$ ausgegeben.
Das auch die Sortierung richtig ist, sieht man anhand der Tatsache, dass immer der kleinstmögliche Startwert gesucht wird.
Also ist auch dieser Algorithmus korrekt, bei Aufruf von |cyclesOf| mit $ready = \empty$ werde nämlich die Zyklen $\zeta_1, \dots, \zeta_o$ zurückgegeben.\\

Anschließend zeigen wir die Optimalität vom eigentlichem Algorithmus, die Berechnung der Instruktionen.
Diese machen wir uns klar, indem wir uns erst für eine beliebige Containerkonstellation, also eine beliebige Permutation überlegen, wie der optimale Weg aussehen muss.
Also sei $\pi$ eine beliebige Permutation über $X$ und $Z := \{\zeta_1, \zeta_2, \dots, \zeta_o\}$ die disjunkten Zyklen, die $\pi$ darstellen.
Nun teilen wir diese Zyklen in die Äquivalenzklassen $A_1, \dots, A_a$ auf.
In jeder Äquivalenzklasse sollen - grob gesprochen - nur Zyklen sein, die sich ``überschneiden''.
Wir definieren uns zunächst eine Äquivalenrelation $\between$
Seien $\eta,\theta \in Z$ zwei Zyklen der Länge $k$ bzw. $l$ der Form $e, \eta(e), \dots, \eta^{k-1}(e)$ bzw. $t, \theta(t), \dots, \theta^{l-1}(t)$.
Weiter seien $E := \{e, \eta(e), \dots, \eta^{k-1}(e)$ und $T := \{t, \theta(t), \dots, \theta^{l-1}(t)$, also jeweils die Mengen der ``bewegten'' Elemente der Zyklen.
Sei o.B.d.A. $\min(E) \leq \min(T)$ (Andernfalls vertauschen wir $\eta$ und $\theta$). Dann gilt $\eta \between \theta$ dann, wenn $\max(T) \leq \max(E)$.
Salopp gesagt, gilt $\eta \between \theta$ gdw. $\theta$ nicht ``innerhalb'' von $\eta$ liegt. \\
Nun machen wir uns noch klar, dass $\between$ auch wirklich eine Äquivalenzrelation auf $Z$ ist.
Die Reflexität ist klar, sei $\eta \in Z$ Zyklus und $E := $ alle Elemente von $\eta$ (die nicht fest gelassen werden).
Dann gilt $\min(E) \leq \min(E)$, also gilt $\eta \between \eta$.\\
Die Symmetrie ist ebenfalls sehr anschaulich, da wir o.B.d.A. $\min(E) \leq \min(T)$ angenommen haben,
können wir ebenfalls o.B.d.A. annehmen $\min(T) \leq \min(E)$ und $\between$ wäre nach Definition somit symmetrisch.\\
Also zeigen wir folgend die Transitivität.
Seien $\eta, \theta, \iota \in Z, \eta \neq \theta \neq \iota$ und es gelte $\eta \between \theta$ und $\theta \between \iota$.
Seien $E,T,I$ die Mengen der Zyklen $\eta, \theta, \iota$.
Dann gilt nach Definition von $\between$ $\min(E) \leq \min(T) \leq \min(I) \wedge \max(E) \geq \max(T) \geq \max(I)$, also gilt auch $E \between I$.
Die Relation $\between$ ist also eine Äquivalenzrelation.\\
Was sagen uns jetzt aber die Äquivalenzklassen? Wir erinnern uns, dass die Zyklen einer Äquivalenzklasse sich ``überschneiden''.
Das heißt, gibt es zu einer Menge Zyklen mehr als eine Äquivalenzklasse, gibt es wohl Zyklen die sich nicht ``überschneiden''.\\
Betrachtet wir nun nochmals den zunächst gewählten Problemlösungsansatz, nämlich beginnend beim ``ersten'' Zyklus alle Zyklen zu bearbeiten und bei Überschneidungen zu unterbrechen.
Aber wir haben gerade erst gezeigt, dass es eben auch Zyklen gibt, die sich \emph{nicht} Überschneiden!
Also muss auch dies (Im Code wäre dies Fall (1) in der Hilfsmethode |step| von |computeCycle|) beachtet werden.\\
Nun formuliere ich einen Satz über den optimalen Weg von Zyklen.\\
Der im Sinne der Aufgabenstellung \emph{optimale Weg} $w$ zu einer die Containerpositionen beschreibenden Permutation $pi$ mit Länge $n$,
die durch die disjunkten Zyklen $Z$ dargestellt werden kann ist folgender.
Sei $a$ die Anzahl der Äquivalenzklassen, in die $Z$ durch $\between$ aufgeteilt wird.
Dann ist der optimale Weg $w = a \cdot 2 + \sum_{i = 1}^{n}{\lvert i-\pi(i)\rvert}$.\\
Dies machen wir uns wie folgt klar. Zunächst betrachte man den Ausdruck $a \cdot 2$.
Da \emph{kein} Container über diese ``Grenze'' gebracht werden muss, aber trotzdem der Kran mindestens einmal zu jeder Position gebracht werden muss, sind hier Leerfahrten nötig.
Genauer gesagt sind \emph{zwei} Leerfahrten nötig, da der Kran (mindestens) einmal hinüber und einmal zurück gebracht werden muss.
Zurück deswegen, weil er zum Schluß auf jeden Fall an der ersten Position wieder angelangen soll.
Folgend setzen wir $w_s := \sum_{i = 1}^{n}{\lvert i-\pi(i)\rvert}$
Nach meiner Behauptung muss also die Summe der Wege innerhalb der Äquivalenzklassen genau gleich $w_s$ sein.
Da immer nur \emph{ein} Container an der Position $c$ auf einmal transportiert werden kann,
und zwar jeweils von $c$ nach $\pi(c)$ ist dieser Weg auf jeden Fall zurückzulegen, also muss $w$ auf jeden Fall größer oder gleich $w_s$ sein.
Also genügt folgend zu betrachten, dass die Summe $w_a$ der minimalen Wege innerhalb jeder Äquivalenzklasse maximal $w_s$ ist.
Angenommen, dies wäre nicht so, also $w_a > w_s$.
Dann müsste es neben den Fahrten vom Containerstellplatz zu der dazugehörigen Wagenpositionen noch mindestens eine weitere Fahrt von $x$ nach $y$ geben,
auf der kein Container ``in die richtige Richtung'' transportiert wird.
Also entweder kein Container (->Leerfahrt) oder aber ein Container der eigentlich von einem Ort $o >= x$ zum Wagen $i <= x$ gebracht werden muss.
Da jedoch alle Zyklen innerhalb eine Äquivalenzklasse ohne Leerfahrt abgearbeitet werden können,
sind neben den $a \cdot 2$ Leerfahrten zwischen Äquivalenzklassen keine weiteren Leerfahrten nötig.
Auch sind keine Fahrten in ``falsche'' Richtungen innerhalb einer Äquivalenzklasse nötig, da alle Zyklen an einem Stück abgearbeitet werden können. %TODO!
Ein Container $i$ muss zudem nie über die Grenzen einer Äquivalenzklasse transportiert werden, da $\pi(i)$ auf jeden Fall in den selben Grenzen liegt.
%TODO: Show that all cycles in one A can be processed at once, w/o empty drives (?)

Nun muss noch gezeigt werden, dass der Algorithmus alle Vorraussetzungen erfüllt und einen optimalen Weg liefert.
Dafür müssen lediglich folgende drei Aspekte gezeigt werden.
\begin{enumerate}
 \item Der erzeugte Weg ist ein zusammenhängender Weg. (Keine Sprünge)
 \item Es werden alle Container an die richtige Position gebracht.
 \item Es werden nie ``unnötige'' Fahrten erzeugt.
\end{enumerate}
Das jeweils die richtige Instruktionen zum Drehen des Krankopfes, Ablegen und Aufnehmen von Containern erzeugt werden, wird hier nicht bewiesen.
Es geht hier ausschließlich um den optimalen Weg.

%TODO:!

\paragraph{Laufzeitverhalten}
Zunächst wird das Laufzeitverhalten des Algorithmus zum Finden der Zyklen analysiert.
|cyclesOf| berechnet in jedem Schritt den neuen Startwert |start|.
Dazu wird die Folge 1 bis zur Permutationslänge $n$ traversiert bis ein Wert gefunden wird der noch nicht abgearbeitet - sprich in |ready| enthalten - ist.
Nimmt man an, dass das Prüfen auf Enthaltensein konstanten Zeitaufwand darstellt (beispielsweise bei Verwendung eines |HashSet|s), dann ergibt dies insgesamt eine Komplexität von $O(n)$.
Die Berechnung eines Zyklus benötigt höchstens die Traversierung der Permutation, also ebenfalls $O(n)$. %TODO: Genauer!
Anschließend werden die Zahlen, die im Zyklus enthalten sind, in |ready| eingefügt. Unter Annahme, dass wiederum ein |HashSet| verwendet wird, ergibt das eine Komplexität von $O(n)$.
Anschließend erfolgt der rekursive Aufruf. Sei $c$ die Anzahl der Zyklen, dann wird |cyclesOf| $c$-mal aufgerufen.
Die Laufzeitkomplexität zur Finden der Zyklen ist also $O(c \cdot n)$.

\subsubsection{Ergebnis und Laufzeitoptimaler Algorithmus} %Erweiterung!
Das Laufzeitverhalten von $O(c \cdot n)$ ist zwar bereits recht gut, da die Anzahl der Zyklen im Normalfall nicht linear mit $n$ steigen.
(Eine zufällig erzeugte Permutation mit $10^7$ Elementen hat meist weniger als 20 Zyklen)% TODO: Beleg
Der Worstcase bei $n/2$ Zyklen führt jedoch zu einer Worstcase-Komplexität von $O(n^2)$. \\
Deshalb soll als Erweiterung die Laufzeitkomplexität weiter verringert werden. \\
Außerdem sind die Algorithmen, wie sie oben angegeben sind, nicht tail-recursive.
Das heißt bei jedem rekursivem Aufruf wird ein neuer Stack-frame allokiert.
In der Praxis heißt dies, dass nur eine Rekursionstiefe von höchstens 10000 möglich ist.
Die Entwicklung eines Laufzeitoptimalen Algorithmus betrachte ich als Erweiterung im Sinne der Allgemeinen Hinweise in den Aufgaben.
Diese ist sinnvoll, denn - wie später in \ref{scalability} gezeigt -
lassen sich damit zu einer Permutation (die die Container darstellt)
mit einer Länge in der Größenordnung $10^7$ innerhalb weniger Minuten Instruktionen berechnen, die einen optimalen Weg liefern.

\paragraph{Verbesserung}
\label{fast_cycler}
Die Verbesserung - und Schwierigkeit - besteht darin, den bisherigen limitierenden Faktor, nämlich die Berechnung der Zyklen zu optimieren.
Außerdem müssen alle rekursiven Funktionen umgeschrieben werden, so dass sie vom Scala compiler tail-call optimiert werden können.
Das heißt, alle rekursiven Aufrufe einer Funktion müssen der letzte Befehl einer Funktion sein.
\begin{lstlisting}
@tailrec
def cyclesOf(ready: ListBuffer[Cycle], perm: Seq[Int],
             handled: Array[Boolean], start: Int = 1): Cycles =
  if(start > perm.length) ready.toList
  else {
    val aCycle = cycle(perm, start)
    for (i <- aCycle) { handled(i-1) = true }
    (start to perm.length) find (i => !(handled(i-1))) match {
      case Some(next) =>
        cyclesOf(ready += aCycle, perm, handled, next)
      case None =>
        (ready += aCycle).toList
    }
  }
\end{lstlisting}


\paragraph{Optimale Ergebnisse}
Wie oben (in \ref{opt_res}) bereits gezeigt, können aus korrekten, sortierten Zyklen Instruktionen, die einen optimalen Weg für den Kran liefern, berechnet werden.
Deshalb muss hier lediglich noch gezeigt werden, dass der neue Algorithmus wiederum korrekte und sortierte Zyklen berechnet. \\

\paragraph{Optimale Laufzeitkomplexität}
... \\
Da jeder Container auf einen Waggon gebracht werden muss, muss für jeden Container mindestens ein Befehl erzeugt werden.
Bei $n$ Container sind dies also $n$ Befehle. Das setzt einen Algorithmus mit einer Laufzeitkomplexität von mindestens $O(n)$ voraus.
Der erstellte Algorithmus hat also \textbf{optimale Laufzeitkomplexität}.

\paragraph{Mögliche Parallelisierung}
Es wurden Überlegungen zur Parallelisierung des Algorithmus zur Berechnung der Instruktionen gemacht.
Aus Zeigründen wurde jedoch auf eine Implementierung verzichtet.
Der Algorithmus kann parallelisiert werden,
indem zunächst für jeden Zyklus die Instruktionsketten berechnet werden und diese nachträglich kombiniert werden.

\subsection{Implementierung}
Die Implementierung gliedert sich folgendermaßen.
\begin{description}
 \item[cycler] Algorithmen zur Berechnung der Zyklen (Sowohl langsamerer, als auch schnellerer)
 \item[Instructor] Algorithmus zur Berechnung der Instruktionen aus den Zyklen
 \item[Gleis] Datenstruktur zur Verwaltung der Containerstellplätze und Waggons
 \item[Maschine] Klasse zur Simulation einer Maschine
 \item[ListBuffer] Modifizierte Variante des standardmäßigem Scala ListBuffer
 \item[Utils] hilfreiche Methoden, u.a. zur Ausgabe in Dateien
\end{description}

\subsubsection{Cycler - Berechnung der Zyklen}
Da beide Algorithmen zur Zyklenfindung implementiert werden sollen, wurde zunächst das |trait| |Cycler| implementiert,
welches die einzige Methode des Moduls |cyclesOf(Seq[Int]):List[List[Int]]| definiert.
Diese soll zu einer gegebenen Permutation eine Liste von nach Startelementen sortierte Zyklen zurückgeben. \\
Die Implementierung des |SlowCycler| erfolgte wie in \ref{slow_cycler}, die des |FastCycler| nach \ref{fast_cycler}.

\subsubsection{Instructor - Berechnung der Instruktionen}
Anschließend wurden im Modul |Instructor| Funktionen zur Berechnung der Instruktionen erstellt.
Diese gliedern sich in die von ``außen'' zu benutzenden Funktionen sowie die ``innen'' benötigten Hilfsfunktionen.
Von außen sind |compute(Seq[Int], Cycler): Seq[Instruction]| und |computeFromCycles(List[List[Int]]): Seq[Instruction]| zu benutzen.
Die letztere berechnet die Liste der Instruktionen aus (meist vorher berechneten) Zyklen,
während die erstere die Benutzung dadurch vereinfacht, nur die Permutation angeben zu müssen (die Zyklen werden dann automatisch berechnet).
Die ``inneren'' Hilfsfunktionen sind folgende. \\
|computeCycle(List[Int], List[List[Int]]): (ListBuffer[Instruction], List[List[Int]])|
gibt zu einem zu bearbeitenden Startzyklus und restlichen Zyklen eine Liste von Instruktionen und eine Liste von unbearbeiteten Zyklen zurück.

\subsubsection{Gleis - Speichern des Zustand}
\label{gleis}
Die Datenstruktur zum Speichern des aktuellen Status der Container, Containerstellplätzen und Waggons wird in der Klasse |Gleis| implementiert.
Ein |Gleis| verwaltet zwei Arrays der Länge $n$. Das erste Array |con| speichert die jeweilige Containernummer auf dem zugehörigen Containerstellplatz.
Das andere Array |wag| speichert die jeweilige Nummer des Container auf einem Waggon.
Zu Beginn wird das Array |con| mit der Permutation initialisiert. \\
Ein |Gleis| stellt die Methoden |takeCon(Int): Int|, |takeWag(Int): Int|, |putCon((Int, Int)): Int| und |putWag((Int, Int)): Int|.
Außerdem wurde die |toString: String| Methode überschrieben, um eine formatierte Ausgabe zu erhalten.
Die oben genannten Methoden sind zur Manipulation der Containernummern zu den jewiligen Containerstellplätzen, bzw. Waggons da.
Genauere Verwendung wird bei späterer Referenz genauer beschrieben.

\subsubsection{Maschine - Interpretieren der Instruktionen}
Um die erzeugten Instruktionen interpretieren zu können, wurde die Klasse |Maschine| geschrieben.
Diese stellt eine Methode |interpret| dar, die eine Befehlskette ausführt.
Eine |Maschine| bedient sich einem |Gleis| um den Zustand zu speichern.
Außerdem wurde die Klasse so gestaltet, dass Unterklassen leicht geschrieben werden können, um beispielsweise eine echte Kransteuerung anzubinden.

\subsubsection{ListBuffer - Erweiterung einer Standardklasse}
Um die Befehlsketten effizient erstellen zu können wird eine Datenstruktur benötigt,
auf der das Anhängen einer zweiten Befehlskette in konstanter Zeit implementiert werden kann.
Anschließend muss sie beginnend bei dem zuerst eingefügtem Element der Einfügereihenfolge folgend in linearer Zeit traversierbar sein.
Diese Bedingungen erfüllt - leider - keine Standardklasse aus der Scala Collections API.
Deswegen wurde die Klasse |ListBuffer| um das Anhängen eines zweiten |ListBuffer|s mit konstantem Zeitaufwand erweitert.

\subsubsection{Utils - Helfende Methoden}
Weitere Methoden, die nützlich im Rahmen der Nutzung des Programmes sind, jedoch nicht direkt zur Implementierung der Aufgabelösung dienen,
wurden in das Modul |Utils| ausgelagert. Besondere Bedeutung hat die Funktion |randPerm|, die zu einer gegebenen Permutationslänge
eine zufällige Permutation berechnet. %TODO: Cite where alg came from?
Außerdem wurden auch Methoden zum Speichern der Instruktionsketten und Permutationen implementiert.

\newpage
\subsection{Programmabläufe}
\paragraph{Beispiel aus der Aufgabenstellung}
Folgend ist der Ablauf der sich bei Eingabe des Beispiels aus der Aufgabestellung ergibt dargestellt. \\
Zunächst wird die Permutation erzeugt und in |perm| gespeichert.
\begin{lstlisting}
scala> val perm = Seq(4,3,2,1)               
perm: IndexedSeq[Int] = WrappedArray(4, 3, 2, 1)
\end{lstlisting}
Anschließend werden die Instruktionen erzeugt und in |instrs| gespeichert.
\begin{lstlisting}
scala> val instrs = Instructor compute perm
instrs: Seq[de.voodle.tim.bwinf.container.Instruction] =
  List(TakeCon, MoveRight(1), Rotate, TakeCon, Rotate, PutCon,
       Rotate,  MoveRight(1), Rotate, PutWag,
       TakeCon, MoveLeft(1),  Rotate, PutWag,
       TakeCon, MoveRight(2), Rotate, PutWag,
       TakeCon, MoveLeft(3),  Rotate, PutWag, TakeCon)
\end{lstlisting}
Nun wird eine |Maschine| erzeugt, die die Instruktionen ausführen kann.
\begin{lstlisting}
scala> val maschine = new Maschine(new Gleis(perm),true)
maschine: de.voodle.tim.bwinf.container.Maschine = 
Container: 4 3 2 1
Waggons:   _ _ _ _
\end{lstlisting}
Zuletzt soll die |Maschine| die Instruktionen interpretieren.
\begin{lstlisting}
scala> maschine interpret instrs                        
1 2 3 4;(m=8)
4 3 2 1;(l=8)
-->	(1)
  -->	(1)
  <--	(1)
  ---->	(2)
<------	(3)
res0: de.voodle.tim.bwinf.container.Gleis = 
Container: _ _ _ _
Waggons:   1 2 3 4
\end{lstlisting}
Bemerkenswert ist hier, dass der erstellte Algorithmus in diesem Fall exakt den gleichen Weg liefert wie im Beispiel der Aufgabenstellung angegeben.
Es gibt noch verschiedene andere Wege.
Beispielsweise kann das Prüfen auf überlappende Zyklen erst beim Zurückfahren erfolgen.
Andere Möglichkeiten für einen optimalen Weg wären folgend dargestellte Abläufen.
Es gibt also insgesamt vier verschieden Fahrpläne, die für das Beispiel einen optimalen Weg ergeben.
\begin{lstlisting}                   
1 2 3 4;(m=8)	1 2 3 4;(m=8)	1 2 3 4;(m=8)
4 3 2 1;(l=8)	4 3 2 1;(l=8)	4 3 2 1;(l=8)
------>	(3)	------>	(3)	---->	(2)
    <--	(1)	  <----	(2)	  <--	(1)
  <--	(1)	  -->	(1)	  -->	(1)
  -->	(1)	  <--	(1)	    -->	(1)
<----	(2)	<--	(1)	<------	(3)
\end{lstlisting}
\newpage
\paragraph{Zufällig erzeugte Permutation}
Ein nächstes - etwas größeres Beispiel ergibt sich aus zufälliger Erzeugung einer Permutation der Länge 20.
Hierbei wird die Hilfsfunktion |randPerm| des Moduls |Utils| aufgerufen und das Ergebnis wie vorher in |perm| gespeichert.
\begin{lstlisting}
scala> val perm = Utils randPerm 20
perm: IndexedSeq[Int] =
  WrappedArray(20, 11, 2, 8,  1, 16, 10, 17, 19, 14,
                5, 12, 9, 3, 13, 15, 18,  4,  7,  6)
\end{lstlisting}
Anschließend werden wieder die Instruktionen mit der Funktion |compute| des Moduls |FastAlgorithm| berechnet und in |instrs| gespeichert.
\begin{lstlisting}
scala> val instrs = Instructor compute perm
instrs: Seq[de.voodle.tim.bwinf.container.Instruction] =
   List(TakeCon, MoveRight(3),  Rotate, TakeCon, Rotate, PutCon,
	Rotate,  MoveRight(4),  Rotate, PutWag,
	TakeCon, MoveRight(4),  Rotate, TakeCon, Rotate, PutCon,
	Rotate,  MoveLeft(0),   Rotate, PutWag,
	TakeCon, MoveRight(5),  Rotate, PutWag,
	TakeCon, MoveRight(1),  Rotate, PutWag, ...
\end{lstlisting}
Zuletzt wird wieder eine |Maschine| |maschine| erzeugt um die Instruktionen zu interpretieren.
\begin{lstlisting}
scala> val maschine = new Maschine(new Gleis(perm),true)      
maschine: de.voodle.tim.bwinf.container.Maschine = 
Container: 20 11 2 8 1 16 10 17 19 14 5 12 9 3 13 15 18 4 7 6
Waggons:    _  _ _ _ _  _  _  _  _  _ _  _ _ _  _  _  _ _ _ _

scala> maschine interpret instrs
 1  2  3  4  5  6  7  8  9 10 11 12 13 14 15 16 17 18 19 20;(m=138)
20 11  2  8  1 16 10 17 19 14  5 12  9  3 13 15 18  4  7  6;(l=138)
--------->                                                 (3)
         ------------>                                     (4)
                     ------------>                         (4)
                                 <                         (0)
                                 --------------->          (5)
                                                --->       (1)
         <------------------------------------------       (14)
         ------------------------------------------------> (16)
               <------------------------------------------ (14)
               ------------------------------>             (10)
                                          <---             (1)
                                    <------                (2)
                        <------------                      (4)
                        ------------------------------>    (10)
                  <------------------------------------    (12)
                  --------->                               (3)
                           ------------>                   (4)
      <---------------------------------                   (11)
   <---                                                    (1)
   --------------------------->                            (9)
            <------------------                            (6)
<------------                                              (4)
res0: de.voodle.tim.bwinf.container.Gleis = 
Container: _ _ _ _ _ _ _ _ _  _  _  _  _  _  _  _  _  _  _  _
Waggons:   1 2 3 4 5 6 7 8 9 10 11 12 13 14 15 16 17 18 19 20
\end{lstlisting}
Permutationen bis zu einer Länge von 20 können wie gezeigt problemlos in der Konsole angezeigt und dargestellt werden.
Durch das gewählte - and die Aufgabenstellung angelehnte - Ausgabeformat können auch die zu fahrende Wege gut in der Konsole dargestellt werden.
Die Optimalheit des Weges kann leicht nachvollzogen werden.
In der ersten Zeile ist die anhand der Permutations ausgerechnete mindestens benötigte Weglänge $m$ ausgegeben.
In der zweiten Zeile ist die anhand der Instruktionen berechnete Weglänge $l$ ausgegeben. Wie zu sehen, stimmen diese überein.
\paragraph{Demonstration der Skalierbarkeit}
\label{scalability}
Nun soll die Skalierbarkeit demonstriert werden,
die als Erweiterung in Form von Tail-rekursiven Funktionen und linearer Laufzeitkomplexität implementiert wurde.\\
Hierfür erzeugen wir eine zufällige Permutation von 6,4 Millionen ($6,4 \cdot 10^6$) Zahlen, die unsere Container darstellt.
Anschließend werden wie oben auch, die Instruktionen berechnet und interpretiert.
Für Demonstrationszwecke wird außerdem die benötigte Zeit für jeden Schritt berechnet.
Dies hat nicht das Ziel genaue Benchmarkwerte zu liefern, sondern vielmehr einen Anhaltspunkt für das Laufzeitverhalten darzustellen.
Hierfür wurde ein kleines Scala Programm geschrieben welches im Modul |Utils| zu finden ist.
\begin{lstlisting}
scala> val verified = Utils demonstrate 6400000
Time used for computing Cycles: 30093
Number of cycles: 18
Time used: 110879
Time used interpreting: 10639
verified: Boolean = true
\end{lstlisting}
Interessant ist hier die Beobachtung, dass es nur 18 Zyklen gibt, bei einer Permutationslänge von $~10^7$.
Insgesamt wurden 110879 Millisekunden, also 110 Sekunden bzw. knapp 2 Minuten benötigt, um die Instruktionen zu berechen.
Dies ist ein Indiz auf oben bewiesene gute Laufzeitkomplexität.
Nach der Berechnung der Instruktionen wurden diese testweise interpretiert. Hierfür wurden knapp 11 Sekunden benötigt.
Zum Schluß wurde außerdem verifiziert, dass jeder Container auf der richtigen Position ist.

\subsection{Programmnutzung}
\lstset{basicstyle=\ttfamily\small}
Die Nutzung des Programms erfolgt primär über eine Scala Console mit richtig eingestelltem Classpath.
Um dies einfach zu erreichen, empfehle Ich Ihnen, im Programmordner Aufgabe2/dist/ die Konsole des Buildprogramm sbt mit |./sbt console| zu starten.
Anschließend sollten Sie zunächst alle Klassen und Module aus dem Paket |de.voodle.tim.bwinf.container| importieren. Dies lässt sich beispielsweise wie folgt machen.
\begin{lstlisting}
scala>import de.voodle.tim.bwinf.container._
import de.voodle.tim.bwinf.container._
\end{lstlisting}
\subsubsection{Permutationen erzeugen}
Permutationen erzeugen Sie entweder durch direkte Eingabe oder Sie lassen eine randomisierte Permutation für eine gegebene Länge erzeugen.
Um eine Permutation direkt einzugeben können Sie einfach die Hilfsfunktionen der Scalabibliothek benutzen.
Speichern Sie einfach das Bild der Permutation in einer |Seq|. Die Permutation aus der Aufgabenstellung geben Sie beispielsweise wie folgt ein.
\begin{lstlisting}
scala>val perm = Seq(4,3,2,1)
perm: Seq[Int] = List(4, 3, 2, 1)  
\end{lstlisting}
Zufällige Permutationen erzeugen Sie mit der Methode |randPerm| im Modul |Utils| unter Angaben einer Länge.
Um eine zufällige Permutation der Länge 4 zu generieren, gehen Sie z.B. wie folgt vor.
\begin{lstlisting}
scala>val perm = Utils randPerm 4                                                                                                 
perm: scala.collection.mutable.IndexedSeq[Int] = WrappedArray(4, 2, 1, 3)  
\end{lstlisting}
\subsubsection{Erzeugen der Instruktionen}
Nachdem Sie nun eine Permutation erzeugt haben, können Sie die Methode |compute| des Moduls |Instructor| verwenden, um die Instruktionen zu berechnen.
\begin{lstlisting}
scala> val instrs = Instructor compute perm
instrs: Seq[de.voodle.tim.bwinf.container.Instruction] = List(TakeCon, ...
\end{lstlisting}
Sie können auch - wenn Sie wollen - zunächst die Zyklen berechnen, mit dem schnellerem |FastCycler| oder mit dem langsameren |SlowCycler|.
Hierzu rufen Sie einfach die Methode |cyclesOf| auf. Z.B. wie folgt.
\begin{lstlisting}
scala> val cycles = FastCycler cyclesOf perm
cycles: de.voodle.tim.bwinf.container.Cycler.Cycles = List(List(1, 4), List(2, 3))

scala> val cycles = SlowCycler cyclesOf perm
cycles: de.voodle.tim.bwinf.container.Cycler.Cycles = List(List(1, 4), List(2, 3))
\end{lstlisting}
Anschließend können Sie die Instruktionen auch direkt aus den Zyklen berechnen. Dafür ist die Methode |computeFromCycles| im Modul |Instructor| da.
\begin{lstlisting}
scala> val instrs = Instructor computeFromCycles cycles
instrs: Seq[de.voodle.tim.bwinf.container.Instruction] = List(TakeCon, ...
\end{lstlisting}
\subsubsection{Simulation der Maschine}
Nun haben Sie bereits die Instruktionskette erzeugt.
Am einfachsten ist es, eine |Maschine| zu erzeugen, diese die Instruktionen ausführen zu lassen und anschließend die Ausgabe zu betrachten. \\
Erzeugen der |Maschine|:
\begin{lstlisting}
scala> val maschine = new Maschine(new Gleis(perm),true)
maschine: de.voodle.tim.bwinf.container.Maschine = 
Container: 4 3 2 1
Waggons:   _ _ _ _
\end{lstlisting}
Ausführen der Instruktionen:
\begin{lstlisting}
scala> maschine interpret instrs
1 2 3 4;(m=8)
4 3 2 1;(l=8)
-->     (1)
  -->   (1)
  <--   (1)
  ----> (2)
<------ (3)
res1: de.voodle.tim.bwinf.container.Gleis = 
Container: _ _ _ _
Waggons:   1 2 3 4
\end{lstlisting}
\subsubsection{Zeitmessung}
Wenn Sie sich zusätzlich noch die Skalierbarkeit nachvollziehen wollen, fordere ich Sie auf die Funktion |demonstrate| im Modul |Utils| auszuprobieren.
Um beispielsweise für 100000 Container Instruktionen ausführen zu lassen und anschließend verifizieren zu lassen, ob auch jeder Container am richtigen Platz angekommen ist, führen Sie folgende Befehle aus.
\begin{lstlisting}
scala> val verified = Utils demonstrate 100000 
Time used for computing Cycles: 707
Number of cycles: 12
Time used: 852
Time used interpreting: 82
Verifying results...
verified: Boolean = true
\end{lstlisting}

\begin{flushright}
\begin{footnotesize}
\emph{Bemerkung:} Die Ausgaben der Konsole wurden per Hand nachformatiert zwecks besserer Einbettung in den Textfluss.
\end{footnotesize}
\end{flushright}

\newpage
\subsection{Programmtext}
Alle Quelldateien finden sich auf der CD unter Aufgabe2/src/
\subsubsection{Cycler}
\lstset{basicstyle=\ttfamily\scriptsize}
\begin{lstlisting}
package de.voodle.tim.bwinf.container
import annotation.tailrec
import scala.collection.mutable.tim.ListBuffer // <-- custom ListBuffer

object Cycler {
  type Cycle = List[Int]
  type Cycles = List[List[Int]]
}
import Cycler._
trait Cycler extends Function1[Seq[Int], List[List[Int]]] {
  def apply(perm: Seq[Int]) = cyclesOf(perm)
  def cyclesOf(perm: Seq[Int]): Cycles
}
object SlowCycler extends Cycler {
  def cycle(perm: Seq[Int], start: Int): Cycle = {
    def step(idx: Int): Cycle = // Hilfsfunktion
      if(start == idx)
        Nil
      else
        idx :: step(perm(idx - 1))
    start :: step(perm(start-1))
  }
  def cyclesOf(perm: Seq[Int]): Cycles = cyclesOf(perm, Set())
  def cyclesOf(perm: Seq[Int], ready: Set[Int]): Cycles =
    (1 to perm.length) find (i => !ready.contains(i)) match {
      case Some(start) =>
        val newCycle = cycle(perm, start)
        val newReady = ready ++ newCycle // O(n)
        newCycle :: cyclesOf(perm, newReady)
      case None =>
        Nil
    }
}
object FastCycler extends Cycler {
  /** Return the list of disjunct cycles sorted ascending by cycle.head */
  def cyclesOf(perm: Seq[Int]): Cycles =
    cyclesOf(new ListBuffer[Cycle], perm, new Array[Boolean](perm.length))

  @tailrec private def cyclesOf(ready: ListBuffer[Cycle], perm: Seq[Int],
                                handled: Array[Boolean], start: Int = 1): Cycles = // c *
    if(start > perm.length || start < 1) ready.toList
    else {
      val aCycle = cycle(perm, start) // O(n_c)
      for (i <- aCycle) { handled(i-1) = true }
      (start to perm.length) find (i => !(handled(i-1))) match { // O(i_c)
        case Some(next) =>
          cyclesOf(ready += aCycle, perm, handled, next)
        case None =>
          (ready += aCycle).toList // O(1)
      }
    }
  /** Small helper function, finding one cycle. */
  private def cycle(perm: Seq[Int], start: Int): Cycle = {// O(n_c)
    @tailrec def step(ready: ListBuffer[Int], idx: Int): Cycle = // O(n_c)
      if(start == idx)
        ready.toList // O(1)
      else
        step(ready += idx,perm(idx - 1))
    (start :: step(new ListBuffer[Int], perm(start - 1)))
  }
}
\end{lstlisting}

\subsubsection{Instructor}
\begin{lstlisting}
package de.voodle.tim.bwinf.container
import scala.annotation.tailrec
import scala.collection.mutable.tim.ListBuffer
import Cycler._ // import types.

object Instructor {
  def compute(perm: Seq[Int], cycler: Cycler = FastCycler): Seq[Instruction] =
    computeFromCycles(cycler cyclesOf perm)
  def computeFromCycles(cycles: Cycles): Seq[Instruction] =
    TakeCon :: computeCycle(cycles.head, cycles.tail)._1.toList

  /**
   * Should be called, after a TakeCon!
   * When a cycle starts, all the containers in the cycles are supposed to be on the
   * container side.
   * Container are always transported on the Container side!
   */
  private def computeCycle(cycle: Cycle, other: Cycles): (ListBuffer[Instruction], Cycles) = {
    val max = cycle.max // O(n_c)

    type Step = (ListBuffer[Instruction], Cycles, Int)
    @tailrec def step(instrs: ListBuffer[Instruction], cyclesLeft: Cycles,
                      prev: Int, cur: Int): Step =
      cyclesLeft.headOption match { // Does another Cycle begins between prev and cur?
        case Some(nextCycle @ (next :: _)) if prev == max && max+1 == next => // (1)
          val (cycleInstrs, newCyclesLeft) =
            computeCycle(cyclesLeft.head, cyclesLeft.tail)
          val extraInstrs = instrs ++=
            ListBuffer(PutCon, MoveRight, TakeCon) ++=
            cycleInstrs ++= ListBuffer(MoveLeft, TakeCon)
          step(extraInstrs, newCyclesLeft, prev, cur)
        case Some(nextCycle @ (next :: _)) if next < cur => // (2)
          val (transInstrs, newCyclesLeft) = computeCycle(nextCycle, cyclesLeft.tail)
          // Move from prev to nextCycle.head (next)
          val newInstrs = instrs ++=
            ListBuffer(Move(prev -> next), Rotate, TakeCon, Rotate, PutCon, Rotate) ++=
            transInstrs
          step(newInstrs, newCyclesLeft, next, cur)
        case _ => // (3)
          val newInstrs = instrs ++= ListBuffer(Move(prev -> cur), Rotate, PutWag, TakeCon)
          (newInstrs, cyclesLeft, cur)
      }

    val erster = cycle.head
    val initial = (ListBuffer[Instruction](), other, erster)
    val (instrs, cyclesLeft, last) = (initial /: (cycle.tail :+ erster)) {
        case ((instrs, cyclesLeft, prev), cur) =>
            step(instrs, cyclesLeft, prev, cur)
      }
    (instrs, cyclesLeft)
  }
}
\end{lstlisting}
\newpage
\subsubsection{Gleis}
\begin{lstlisting}
package de.voodle.tim.bwinf.container

class Gleis(initCon: Seq[Int]) {
  val length = initCon.length
  private val con = Seq(initCon: _*).toArray
  private val wag = new Array[Int](length)

  private def arrTake(arr: Array[Int])(i: Int): Int = {
    val res = arr(i-1)
    arr(i-1) = 0
    res
  }
  private def arrPut(arr: Array[Int])(map: (Int, Int)) = map match {
    case (i, what) =>
      require(arr(i-1) == 0, "arr(i-1) at " + i + " must be 0, but is " + arr(i-1))
      arr(i-1) = what
  }

  def takeWag(i: Int) = arrTake(wag)(i)
  def takeCon(i: Int) = arrTake(con)(i)
  def putWag(map: (Int, Int)) = arrPut(wag)(map)
  def putCon(map: (Int, Int)) = arrPut(con)(map)

  private def arrString(arr: Array[Int]) = // Only print first 100
    arr take 100 map (i => if(i == 0) "_" else i.toString) mkString " "
  override def toString =
     "Container: " + arrString(con) + "\n" +
     "Waggons:   " + arrString(wag)

  // Immutable Vector copies!
  def container = Vector(con: _*)
  def waggons = Vector(wag: _*)
}
\end{lstlisting}
\newpage
\subsubsection{Maschine}
\begin{lstlisting}
package de.voodle.tim.bwinf.container
import annotation.tailrec

class Maschine(protected val gleis: Gleis,
                 private val print: Boolean = false) {
  import Maschine._
  private val length = gleis.length
  private val numLength = digits(length)
  private val space = " " * (numLength+1)
  private val arrow = "-" * (numLength+1)

  private def minLength =
    gleis.container.zipWithIndex.map { case (v,i) => ((i+1)-v).abs } sum

  def log(str: =>Any) = if(print) println(str) else ()
  
  def logInts(ints: =>Seq[Int]): String =
    (for(i <- ints) yield {
        val diff = numLength - digits(i)
        " " * diff + i
      }) mkString (" ")

  def interpret(instrs: Seq[Instruction]): Gleis = {
    log(logInts(1 to length) + ";(m=" + minLength + ")")
    log(logInts(gleis.container) + ";(l=" + instrs.map(_.len).sum + ")")
    interpret(instrs.toList,0,0,1)
  }
  // Attach point for further actions (for subclasses)
  protected def act(instrs: List[Instruction]) {}
  
  @tailrec
  private def interpret(instrs: List[Instruction], con: Int, wag: Int, idx: Int): Gleis = {
    act(instrs)
    instrs match { // Recursivly check
      case Rotate :: xs =>
        interpret(xs,wag,con,idx)
      case TakeCon :: xs =>
        interpret(xs, gleis takeCon idx, wag, idx)
      case TakeWag :: xs =>
        interpret(xs, 0, gleis takeWag idx, idx)
      case PutCon :: xs =>
        gleis putCon (idx -> con)
        interpret(xs, 0, wag, idx)
      case PutWag :: xs =>
        gleis putWag (idx -> wag)
        interpret(xs, con, 0, idx)
      case MoveRight(len) :: xs =>
        log(space * (idx-1) + arrow * len + ">" +
            space * (length-len-idx) + " (" + len + ")")
        interpret(xs, con, wag, idx+len)
      case MoveLeft(len) :: xs =>
        log(space * (idx-1-len) + "<" + arrow * len +
            space * (length-idx) + " (" + len + ")")
        interpret(xs, con, wag, idx-len)
      case Nil => gleis // Do Nothing
    }
  }
  override def toString = gleis.toString
}
object Maschine {
  private def digits(num: Int) = (math.log10(num) + 1).floor.toInt
}
\end{lstlisting}
\newpage
\subsubsection{Instructions}
\begin{lstlisting}
package de.voodle.tim.bwinf.container

sealed trait Instruction {
  def len: Int = 0
  def short: String = "" + toString.head
}
case object PutWag extends Instruction
case object PutCon extends Instruction
case object Rotate extends Instruction
case object TakeWag extends Instruction
case object TakeCon extends Instruction
sealed trait Move extends Instruction {
  override def short = (toString filter (_.isUpper)) + "(" + len + ")"
}
object Move {
  def apply(len: Int): Move =
    if(len > 0) MoveRight(len)
    else        MoveLeft(-len)
  def apply(fromTo: (Int, Int)): Move = fromTo match {
    case (from,to) => Move(to - from)
  }
}
case class MoveLeft(override val len: Int) extends Move
object MoveLeft extends MoveLeft(1)
case class MoveRight(override val len: Int) extends Move
object MoveRight extends MoveRight(1)
\end{lstlisting}

\subsubsection{Utils}
\begin{lstlisting}
package de.voodle.tim.bwinf.container

object Utils {
  import scala.util.Random
  import scala.collection.mutable.IndexedSeq
  def randPerm(n: Int) = {
    // Make sure we don't convert it to an WrappedArray to often.
    val a: IndexedSeq[Int] = new Array[Int](n)
    // Init array // O(n)
    for (idx <- 0 until n) a(idx) = idx + 1
    // randomize array // O(n)
    for (i <- n to 2 by -1) {
      val di = Random.nextInt(i)
      val swap = a(di)
      a(di) = a(i-1)
      a(i-1) = swap
    }
    a // return array
  }

  def demonstrate(n: Int) = {
    val startTime = System.currentTimeMillis
    val perm = randPerm(n)
    val cycles = FastCycler cyclesOf perm
    println("Time used for computing Cycles: " + (System.currentTimeMillis - startTime))
    println("Number of cycles: " + cycles.length)
    val instrs = Instructor computeFromCycles cycles
    val endTime = System.currentTimeMillis
    println("Time used: " + (endTime - startTime))
    val gleis = new Gleis(perm)
    val maschine = new Maschine(gleis)
    maschine interpret instrs
    println("Time used interpreting: " + (System.currentTimeMillis - endTime))
    println("Verifying results...")
    gleis.waggons.zipWithIndex forall (xy => xy._1 == xy._2 + 1)
  }
}
\end{lstlisting}
\newpage
\subsubsection{ListBuffer}
\begin{lstlisting}
package scala.collection.mutable.tim
import scala.collection.{mutable,generic,immutable}
import mutable._
import generic._
import immutable.{List, Nil, ::}

/** A `Buffer` implementation back up by a list. It provides constant time
 *  prepend and append. Most other operations are linear.
 *
 *  @author  Tim Taubner
 *  @author  Matthias Zenger
 *  @author  Martin Odersky
 *  @version 2.8.tim
 *  @since   1
 *
 *  @tparam A    the type of this list buffer's elements.
 *
 *  @define Coll ListBuffer
 *  @define coll list buffer
 *  @define thatinfo the class of the returned collection. In the standard library configuration,
 *    `That` is always `ListBuffer[B]` because an implicit of type `CanBuildFrom[ListBuffer, B, ListBuffer[B]]`
 *    is defined in object `ListBuffer`.
 *  @define $bfinfo an implicit value of class `CanBuildFrom` which determines the
 *    result class `That` from the current representation type `Repr`
 *    and the new element type `B`. This is usually the `canBuildFrom` value
 *    defined in object `ListBuffer`.
 *  @define orderDependent
 *  @define orderDependentFold
 *  @define mayNotTerminateInf
 *  @define willNotTerminateInf
 */
@serializable @SerialVersionUID(3419063961353022661L)
final class ListBuffer[A]
      extends Buffer[A]
         with GenericTraversableTemplate[A, ListBuffer]
         with BufferLike[A, ListBuffer[A]]
         with Builder[A, List[A]]
         with SeqForwarder[A]
{
  override def companion: GenericCompanion[ListBuffer] = ListBuffer

  import scala.collection.Traversable

  private var start: List[A] = Nil
  private var last0: ::[A] = _
  private var exported: Boolean = false
  private var len = 0

  protected def underlying: immutable.Seq[A] = start

  /** The current length of the buffer.
   *
   *  This operation takes constant time.
   */
  override def length = len

  // Implementations of abstract methods in Buffer

  override def apply(n: Int): A =
    if (n < 0 || n >= len) throw new IndexOutOfBoundsException(n.toString())
    else super.apply(n)

  /** Replaces element at index `n` with the new element
   *  `newelem`. Takes time linear in the buffer size. (except the
   *  first element, which is updated in constant time).
   *
   *  @param n  the index of the element to replace.
   *  @param x  the new element.
   *  @throws Predef.IndexOutOfBoundsException if `n` is out of bounds.
   */
  def update(n: Int, x: A) {
    try {
      if (exported) copy()
      if (n == 0) {
        val newElem = new :: (x, start.tail);
        if (last0 eq start) {
          last0 = newElem
        }
        start = newElem
      } else {
        var cursor = start
        var i = 1
        while (i < n) {
          cursor = cursor.tail
          i += 1
        }
        val newElem = new :: (x, cursor.tail.tail)
        if (last0 eq cursor.tail) {
          last0 = newElem
        }
        cursor.asInstanceOf[::[A]].tl = newElem
      }
    } catch {
      case ex: Exception => throw new IndexOutOfBoundsException(n.toString())
    }
  }

  // THIS PART IS NEW (by tim8dev):

  /** Appends a single element to this buffer. This operation takes constant time.
   *
   *  @param x  the element to append.
   *  @return   this $coll.
   */
  def += (x: A): this.type = {
    val newLast = new :: (x,Nil)
    append(newLast, newLast, 1)
  }

  override def ++=(xs: TraversableOnce[A]): this.type = xs match {
    case some : ::[A] =>
      append(some, some.last.asInstanceOf[::[A]], some.length)
    case buff : ListBuffer[A] =>
      buff.start match {
        case some : ::[A] =>
          if(buff.exported)
            buff.copy()
          buff.exported = true
          append(some, buff.last0, buff.len)
        case Nil =>
          this
      }
    case xs =>
      super.++=(xs)
  }

  private def append(x: ::[A], last: ::[A], length: Int): this.type = {
    if(exported) copy()
    if(start.isEmpty) {
      last0 = last
      start = x
    } else {
      val last1 = last0
      last1.tl = x
      last0 = last
    }
    len += length
    this
  }

  // END OF NEW PART (by tim8dev).

  /** Clears the buffer contents.
   */
  def clear() {
    start = Nil
    exported = false
    len = 0
  }

  /** Prepends a single element to this buffer. This operation takes constant
   *  time.
   *
   *  @param x  the element to prepend.
   *  @return   this $coll.
   */
  def +=: (x: A): this.type = {
    if (exported) copy()
    val newElem = new :: (x, start)
    if (start.isEmpty) last0 = newElem
    start = newElem
    len += 1
    this
  }

  /** Inserts new elements at the index `n`. Opposed to method
   *  `update`, this method will not replace an element with a new
   *  one. Instead, it will insert a new element at index `n`.
   *
   *  @param  n     the index where a new element will be inserted.
   *  @param  iter  the iterable object providing all elements to insert.
   *  @throws Predef.IndexOutOfBoundsException if `n` is out of bounds.
   */
  def insertAll(n: Int, seq: Traversable[A]) {
    try {
      if (exported) copy()
      var elems = seq.toList.reverse
      len += elems.length
      if (n == 0) {
        while (!elems.isEmpty) {
          val newElem = new :: (elems.head, start)
          if (start.isEmpty) last0 = newElem
          start = newElem
          elems = elems.tail
        }
      } else {
        var cursor = start
        var i = 1
        while (i < n) {
          cursor = cursor.tail
          i += 1
        }
        while (!elems.isEmpty) {
          val newElem = new :: (elems.head, cursor.tail)
          if (cursor.tail.isEmpty) last0 = newElem
          cursor.asInstanceOf[::[A]].tl = newElem
          elems = elems.tail
        }
      }
    } catch {
      case ex: Exception =>
        throw new IndexOutOfBoundsException(n.toString())
    }
  }

  /** Removes a given number of elements on a given index position. May take
   *  time linear in the buffer size.
   *
   *  @param n         the index which refers to the first element to remove.
   *  @param count     the number of elements to remove.
   */
  override def remove(n: Int, count: Int) {
    if (exported) copy()
    val n1 = n max 0
    val count1 = count min (len - n1)
    var old = start.head
    if (n1 == 0) {
      var c = count1
      while (c > 0) {
        start = start.tail
        c -= 1
      }
    } else {
      var cursor = start
      var i = 1
      while (i < n1) {
        cursor = cursor.tail
        i += 1
      }
      var c = count1
      while (c > 0) {
        if (last0 eq cursor.tail) last0 = cursor.asInstanceOf[::[A]]
        cursor.asInstanceOf[::[A]].tl = cursor.tail.tail
        c -= 1
      }
    }
    len -= count1
  }

// Implementation of abstract method in Builder

  def result: List[A] = toList

  /** Converts this buffer to a list. Takes constant time. The buffer is
   *  copied lazily, the first time it is mutated.
   */
  override def toList: List[A] = {
    exported = !start.isEmpty
    start
  }

// New methods in ListBuffer

  /** Prepends the elements of this buffer to a given list
   *
   *  @param xs   the list to which elements are prepended
   */
  def prependToList(xs: List[A]): List[A] =
    if (start.isEmpty) xs
    else { last0.tl = xs; toList }

// Overrides of methods in Buffer

  /** Removes the element on a given index position. May take time linear in
   *  the buffer size.
   *
   *  @param  n  the index which refers to the element to delete.
   *  @return n  the element that was formerly at position `n`.
   *  @note      an element must exists at position `n`.
   *  @throws Predef.IndexOutOfBoundsException if `n` is out of bounds.
   */
  def remove(n: Int): A = {
    if (n < 0 || n >= len) throw new IndexOutOfBoundsException(n.toString())
    if (exported) copy()
    var old = start.head
    if (n == 0) {
      start = start.tail
    } else {
      var cursor = start
      var i = 1
      while (i < n) {
        cursor = cursor.tail
        i += 1
      }
      old = cursor.tail.head
      if (last0 eq cursor.tail) last0 = cursor.asInstanceOf[::[A]]
      cursor.asInstanceOf[::[A]].tl = cursor.tail.tail
    }
    len -= 1
    old
  }

  /** Remove a single element from this buffer. May take time linear in the
   *  buffer size.
   *
   *  @param x  the element to remove.
   *  @return   this $coll.
   */
  override def -= (elem: A): this.type = {
    if (exported) copy()
    if (start.isEmpty) {}
    else if (start.head == elem) {
      start = start.tail
      len -= 1
    } else {
      var cursor = start
      while (!cursor.tail.isEmpty && cursor.tail.head != elem) {
        cursor = cursor.tail
      }
      if (!cursor.tail.isEmpty) {
        val z = cursor.asInstanceOf[::[A]]
        if (z.tl == last0)
          last0 = z
        z.tl = cursor.tail.tail
        len -= 1
      }
    }
    this
  }

  override def iterator = new Iterator[A] {
    var cursor: List[A] = null
    def hasNext: Boolean = !start.isEmpty && (cursor ne last0)
    def next(): A =
      if (!hasNext) {
        throw new NoSuchElementException("next on empty Iterator")
      } else {
        if (cursor eq null) cursor = start else cursor = cursor.tail
        cursor.head
      }
  }

  /** expose the underlying list but do not mark it as exported */
  override def readOnly: List[A] = start

  // Private methods

  /** Copy contents of this buffer */
  private def copy() {
    var cursor = start
    val limit = last0.tail
    clear
    while (cursor ne limit) {
      this += cursor.head
      cursor = cursor.tail
    }
  }

  override def equals(that: Any): Boolean = that match {
    case that: ListBuffer[_] => this.readOnly equals that.readOnly
    case _                   => super.equals(that)
  }

  /** Returns a clone of this buffer.
   *
   *  @return a <code>ListBuffer</code> with the same elements.
   */
  override def clone(): ListBuffer[A] = (new ListBuffer[A]) ++= this

  /** Defines the prefix of the string representation.
   *
   *  @return the string representation of this buffer.
   */
  override def stringPrefix: String = "ListBuffer"
}

/** $factoryInfo
 *  @define Coll ListBuffer
 *  @define coll list buffer
 */
object ListBuffer extends SeqFactory[ListBuffer] {
  implicit def canBuildFrom[A]: CanBuildFrom[Coll, A, ListBuffer[A]] = new GenericCanBuildFrom[A]
  def newBuilder[A]: Builder[A, ListBuffer[A]] = new GrowingBuilder(new ListBuffer[A])
}
\end{lstlisting}
