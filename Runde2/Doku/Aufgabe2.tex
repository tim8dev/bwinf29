\section{Zweite bearbeitete Aufgabe: (2) Containerklamüsel}
\subsection{Lösungsidee}
\subsubsection{Vorüberlegungen}
Die Anordnung der Waggons zu den Container ist eine bijektive Abbildung von $[1,n]$ nach $[1,n]$, sprich, eine Permutation.
Jede Permutation lässt sich als Folge von disjunkten Zyklen darstellen. \\ %TODO: CITE!
Dies erwies sich als günstig, denn nun ist das Problem aufgebrochen in folgende zwei Teile.
Der erste ist, die Container eines Zyklus an die richtige Stelle zu bringen.
Dies lässt sich relativ leicht realisieren, indem der Container am Anfang der Zyklen an die richtige Position gebracht wird, anschließend der - dortige - zweite an die richtige, usw., bis der Ausgangspunkt wieder erreicht ist.
Der zweite - etwas schwierige - Teil besteht darin, die Zyklenabarbeitung dort zu unterbrechen, wo eine andere beginnt. \\
Etwas anders ausgedrückt:
Beginnt man an dem Anfang eines Zyklus, können dessen Container ``in einem Stück'' an die richtige Stelle gebracht werden und der Kran anschließend wieder an der Ausgangsposition angelangen.
Wir werden etwas später sehen, dass dadurch tatsächlich auch immer ein optimalen Weg (zumindest innerhalb eines Zyklus) gefunden werden kann.
Durch entsprechend richtige ``Konkatenation'' der einzelnen Befehlsketten für die einzelnen Zyklen lässt sich ein insgesamt richtiger Algorithmus erstellen.
\subsubsection{Datenstruktur}
Permutationen können in jeder indexierten Liste gespeichert werden.
Da in der Informatik jedoch indexierte Listen (insbesondere Arrays) Indizes aus $[0,n[$ besitzen muss dies beim Zugriff beachtet werden.
Um also die Zahl $p$ zu finden, auf die $i$ durch $perm$ abgebildet wird, gilt $p = perm(i-1)$ und nicht $p = perm(i)$. \\
Es wird außerdem noch eine einfache Datenstruktur benötigt, um das Gleis mit Containerstellplätzen und Waggons abzubilden. \\
Diese ....
% Zyklen
\subsubsection{Ergebnisoptimaler Algorithmus}
\paragraph{Entwurf}
Der Entwurf dieses Algorithmus' ergibt sich aus den obigen Überlegungen. Zunächst werden die disjunkten Zyklen berechnet.
Hierfür wird folgende Hilfsfunktion zur Berechnung \emph{eines} Zyklus' verwendet.
\lstset{language=Scala}
\lstset{basicstyle=\small}
\begin{lstlisting}
def cycle(perm: Seq[Int], start: Int): List[Int] = {
  def step(idx: Int): List[Int] =
    if(start == idx) Nil
    else idx :: step(perm(idx - 1))
  start :: step(perm(start-1))
}
\end{lstlisting}
Salopp gesagt, hangelt man sich so lange - bei einem Startindex beginnend - durch die Permutation, bis man wieder beim Anfangswert ankommt.\\
Nun lässt sich auch recht einfach ein Algorithmus zum finden der disjunkten Zyklen einer Permutation p angeben.
\lstset{language=Scala}
\lstset{basicstyle=\small}
\begin{lstlisting}
def cyclesOf(perm: Seq[Int], ready: Set[Int]): List[List[Int]] =
  if(ready.size == perm.length) Nil
  else {
    val start = (1 to perm.length) find (i => !ready.contains(i))
    val newCycle = cycle(perm, start)
    val newReady = ready ++ newCycle
    newCycle :: cyclesOf(perm, newReady)
  }
\end{lstlisting}

\paragraph{Optimale Ergebnisse}
Dieser Algorithmus liefert bereits optimale Ergebnisse. Um dies zu zeigen, wird zunächst bewiesen, dass die Zyklen richtig gefunden werden. \\
%Dies zeige ich durch einen Gegenteilbeweis. Nehmen wir an, die Zyklen wären nicht richtig.
%Dies würde bedeuten, dass entweder
%\begin{enumerate}
% \item zwei Zahlen innerhalb eines Zyklus' vertauscht sind, oder
% \item nicht alle Zahlen enthalten sind, oder
% \item eine Zahl doppelt enthalten ist. (nicht disjunkt)
%\end{enumerate}
Zunächst wird die Hilfsfunktion |step| der Funktion |cycle| auf Korrektheit geprüft. Da es sich um eine rekursive Funktion handelt, bietet sich hier Induktion an.
Der Beweis, das die Hilfsfunktion |step| zu einer gegebenen Permutation $perm: [0,n[ \rightarrow [1,n]$, einem Start- und Endwert |start| und einem Index |idx|
eine Liste von Zahlen mit Länge n liefert, für die folgendes gilt. ... \\
Induktionsanfang:

... \\
Im folgenden können wir uns also der Korrektheit von |cycle| sicher sein. 
... \\
Anschließend zeigen wir die Optimalheit vom eigentlichem Algorithmus, die Berechnung der Instruktionen.

\paragraph{Laufzeitverhalten}
% O(c*n)
Zunächst wird das Laufzeitverhalten des Algorithmus zum Finden der Zyklen analysiert.
|cyclesOf| berechnet in jedem Schritt den neuen Startwert |start|.
Dazu wird die Folge 1 bis zur Permutationslänge traversiert bis ein Wert gefunden wird der noch nicht abgearbeitet - sprich in |ready| enthalten - ist.
Nimmt man an, dass das Prüfen auf Enthaltensein konstanten Zeitaufwand darstellt (Bsp. bei Verwendung eines |HashSet|s), dann ergibt dies insgesamt eine Komplexität von $O(n)$.
Die Berechnung eines Zyklus benötigt höchstens die Traversierung der Permutation, also ebenfalls $O(n)$. %TODO: Genauer!
Anschließend werden die Zahlen, die im Zyklus enthalten sind, in |ready| eingefügt. Unter Annahme, dass wieder ein |HashSet| verwendet wird, ergibt das eine Komplexität von $O(n)$.
Anschließend erfolgt der rekursive Aufruf. Sei $c$ die Anzahl der Zyklen, dann wird |cyclesOf| $c$-mal aufgerufen.
Die Laufzeitkomplexität zur Finden der Zyklen ist also $O(c \cdot n)$. \\

\subsubsection{Optimaler Algorithmus} %Erweiterung!
Das Laufzeitverhalten von $O(c \cdot n)$ ist zwar bereits recht gut, da die Anzahl der Zyklen im Normalfall nicht linear mit $n$ steigen.
(Eine zufällig erzeugte Permutation mit $10^7$ Elementen hat meist weniger als 20 Zyklen)% TODO: Beleg
Der Worstcase bei $n/2$ Zyklen führt zu einer Worstcase-Komplexität von $O(n²)$. \\
Als Erweiterung soll die Laufzeitkomplexität weiter verringert werden. \\
Außerdem sind die Algorithmen, wie sie oben angegeben sind, nicht tail-recursive.
Das heißt bei jedem rekursivem Aufruf wird ein neuer Stack-frame allokiert.
In der Praxis heißt dies, dass nur eine Rekursionstiefe von höchstens 10000 möglich ist.

\paragraph{Optimale Ergebnisse}
Da oben %TODO: Ref
bereits gezeigt wurde, dass aus korrekten Zyklen Instruktionen, die einen optimalen Weg für den Kran liefern, berechnet werden.
Muss hier nur noch gezeigt werden, dass wiederum korrekte Zyklen berechnet werden. \\

\paragraph{Optimale Laufzeitkomplexität}
... \\
Da jeder Container auf einen Waggon gebracht werden muss, muss für jeden Container mindestens ein Befehl erzeugt werden.
Bei $n$ Container sind dies also $n$ Befehle. Das setzt einen Algorithmus mit einer Laufzeitkomplexität von mindestens $O(n)$ voraus.
Der erstellte Algorithmus hat also \textbf{optimale Laufzeitkomplexität}.

% TODO?: Concurrent Algorithm.
\subsection{Implementation}
\newpage
\subsection{Programmabläufe}
\paragraph{Beispiel aus der Aufgabenstellung}
Das Folgende zeigt den Ablauf, bei Eingabe des Beispiels aus der Aufgabestellung. \\
Zunächst wird die Permutation in |perm| gespeichert.
\begin{lstlisting}
scala> val perm = Seq(4,3,2,1)               
perm: Seq[Int] = List(4, 3, 2, 1)
\end{lstlisting}
Anschließend werden die Instruktionen erzeugt und in |instrs| gespeichert.
\begin{lstlisting}
scala> val instrs = FastAlgorithm compute perm
instrs: Seq[de.voodle.tim.bwinf.container.Instruction] =
  Queue(Take, MoveRight(1), Put, Take, MoveRight(1),
        Swap, MoveLeft(1), Put, Take, MoveRight(2),
        Swap, MoveLeft(3), Put)
\end{lstlisting}
Nun wird eine |Maschine| erzeugt, die die Instruktionen ausführen kann.
\begin{lstlisting}
scala> val maschine = new Maschine(new Gleis(perm),true)
maschine: de.voodle.tim.bwinf.container.Maschine = 
Container: 4 3 2 1
Waggons:   _ _ _ _
\end{lstlisting}
Zuletzt soll die Maschine die Instruktionen interpretieren.
\begin{lstlisting}
scala> maschine interpret instrs                        
4 3 2 1
-->
  -->
  <--
  ---->
<------
res0: de.voodle.tim.bwinf.container.Gleis = 
Container: _ _ _ _
Waggons:   1 2 3 4
\end{lstlisting}


\subsection{Programmtext}
\subsection{Programmnutzung}