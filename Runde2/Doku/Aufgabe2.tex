\section{Zweite bearbeitete Aufgabe: (2) Containerklamüsel}
\subsection{Lösungsidee}
\subsubsection{Vorüberlegungen}
Die Anordnung der Waggons zu den Container ist eine bijektive Abbildung von $[1,n]$ nach $[1,n]$, sprich, eine Permutation.
Jede Permutation lässt sich als Folge von disjunkten Transpositionen darstellen. \\ %TODO: CITE!
Dies erwies sich als günstig, denn nun ist das Problem aufgebrochen in folgende zwei Teile.
Der erste ist, die Container einer Transposition an die richtige Stelle zu bringen.
Dies lässt sich relativ leicht realisieren, indem der Container am Anfang der Transposition an die richtige Position gebracht wird, anschließend der - dortige - zweite an die richtige, usw., bis der Ausgangspunkt wieder erreicht ist.
Der zweite - etwas schwierige - Teil besteht darin, die Transpositionsabarbeitung dort zu unterbrechen, wo eine andere beginnt. \\
Etwas anders ausgedrückt:
Beginnt man an dem Anfang einer Transposition, können alle Container dieser ``in einem Stück'' an die richtige Stelle gebracht werden und anschließend ist man wieder an der Ausgangsposition.
Wir werden etwas später sehen, dass dies tatsächlich immer ein optimalen Weg (zumindest innerhalb einer Transposition) ist.
\subsubsection{Datenstruktur}
Es wurde eine einfache Datenstruktur benötigt, um das Gleis mit Containerstellplätzen und Waggons abzubilden. \\
Diese ....
% Transpositionen
\subsubsection{Algorithmen}
\paragraph{Ergebnisoptimaler Algorithmus}
\subparagraph{Entwurf} Der Entwurf dieses Algorithmus' ergibt sich aus den obigen Überlegungen. Zunächst werden die Transpositionen wie folgt gesucht:
\subparagraph{Optimale Ergebnisse}
\subparagraph{Laufzeitverhalten} % n²
\paragraph{Optimaler Algorithmus}
\subparagraph{Optimale Ergebnisse}

\subparagraph{Optimale Laufzeitkomplexität}

\subsection{Implementation}
